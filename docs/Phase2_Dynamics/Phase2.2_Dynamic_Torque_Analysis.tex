\documentclass[a4paper, 12pt]{article}

% --- PACKAGES ---
\usepackage{fontspec}
\usepackage{polyglossia}
\setmainlanguage{thai}
\setotherlanguage{english}
\defaultfontfeatures{Scale=MatchLowercase}
\setmainfont{TH SarabunPSK}
\newfontfamily\thaifont{TH SarabunPSK}
\usepackage{amsmath} % For math environments
\usepackage{geometry} % For page margins
\geometry{a4paper, margin=1in}
\usepackage{graphicx}
\usepackage{hyperref}
\usepackage{booktabs} % For professional tables
\usepackage{xcolor}
\usepackage{siunitx}
\usepackage{float}

% --- TITLE ---
\title{การวิเคราะห์ทอร์กแบบไดนามิก \\ (Dynamic Torque Analysis) \\ สำหรับกลไก 5-Bar Parallel Linkage}
\author{นายธีรโชติ เมืองจำนงค์}
\date{\today}

% --- DOCUMENT ---
\begin{document}

\maketitle

\section{บทนำ}

เอกสารนี้นำเสนอการวิเคราะห์ทอร์กแบบไดนามิก (Dynamic Torque Analysis) สำหรับกลไก 5-Bar Parallel Linkage ที่ใช้ในหุ่นยนต์สี่ขา 8-DOF โดยคำนึงถึงผลของแรงเฉื่อย (Inertial Effects) แรงโน้มถ่วง (Gravity Forces) และการกระจายน้ำหนักของหุ่นยนต์ทั้งระบบ การวิเคราะห์นี้คำนวณความต้องการทอร์กแบบไดนามิกสำหรับการเดินแบบวิถีรูปไข่ (Elliptical Gait) ที่ความถี่ 1 Hz เพื่อตรวจสอบความเหมาะสมของมอเตอร์ที่เลือก

\subsection{วัตถุประสงค์}

วัตถุประสงค์หลักของการวิเคราะห์นี้คือการคำนวณความต้องการทอร์กแบบไดนามิกสำหรับมอเตอร์แต่ละตัวในกลไก 5-Bar Parallel Linkage ระหว่างการเดินแบบวิถีรูปไข่ การวิเคราะห์นี้ขยายผลจากการวิเคราะห์ทอร์กแบบสถิต (Static Torque Analysis) โดยพิจารณาเพิ่มเติม:

\begin{itemize}
    \item ผลของแรงเฉื่อยจากความเร่งของลิงก์ (Inertial effects from link accelerations)
    \item ส่วนสมทบของโมเมนต์ความเฉื่อยในการหมุน (Rotational inertia contributions)
    \item การกระจายน้ำหนักหุ่นยนต์ทั้งหมด (มวลรวม 6.70 kg)
    \item การเคลื่อนที่แบบไดนามิกที่ความถี่ 1 Hz
\end{itemize}

\subsection{ข้อมูลหุ่นยนต์}

\textbf{ลักษณะหุ่นยนต์:}
\begin{itemize}
    \item \textbf{ประเภท:} หุ่นยนต์สี่ขา 8 องศาอิสระ (8-DOF Quadruped Mobile Robot)
    \item \textbf{กลไกขา:} 5-Bar Parallel Linkage
    \item \textbf{มวลรวม:} 6.70 kg แบ่งเป็น:
    \begin{itemize}
        \item แบตเตอรี่และอิเล็กทรอนิกส์: 2.00 kg
        \item โครงสร้าง/แชสซี: 1.62 kg
        \item มอเตอร์ (8 ตัว): 3.08 kg
    \end{itemize}
    \item \textbf{มวลต่อขา:} 1.675 kg (กระจายเท่ากันทั้ง 4 ขา)
\end{itemize}

\section{พารามิเตอร์กลไก}

\subsection{ขนาดลิงก์}

\begin{table}[H]
\centering
\begin{tabular}{@{}lcc@{}}
\toprule
\textbf{ลิงก์} & \textbf{ความยาว (mm)} & \textbf{คำอธิบาย} \\ 
\midrule
$L_{AC}$ & 105 & ลิงก์มอเตอร์ซ้าย (L1) \\
$L_{BD}$ & 105 & ลิงก์มอเตอร์ขวา (L2) \\
$L_{CE}$ & 145 & ลิงก์เชื่อมซ้าย (L3) \\
$L_{DE}$ & 145 & ลิงก์เชื่อมขวา (L4) \\
$L_{EF}$ & 40 & ออฟเซ็ตปลายเท้า \\
\bottomrule
\end{tabular}
\caption{ขนาดความยาวของลิงก์}
\end{table}

\subsection{ตำแหน่งมอเตอร์}

\begin{align}
P_A &= (-42.5, 0) \text{ mm} \quad \text{(มอเตอร์ซ้าย)} \\
P_B &= (42.5, 0) \text{ mm} \quad \text{(มอเตอร์ขวา)}
\end{align}

\subsection{คุณสมบัติมวล (จาก CAD)}

\textbf{วัสดุ:} PA12-HP Nylon ความหนาแน่น $\rho = 1120$ kg/m³

\begin{table}[H]
\centering
\begin{tabular}{@{}lccccc@{}}
\toprule
\textbf{ลิงก์} & \textbf{มวล} & \textbf{ความยาว} & \textbf{อัตราส่วน COM} & \textbf{ตำแหน่ง COM} & \textbf{$I_{zz}$} \\ 
& (g) & (mm) & (\%) & (mm) & (kg·m²) \\
\midrule
L1 (AC) & 24.88 & 105 & 33.56 & 35.24 & $\approx 0$ \\
L2 (BD) & 35.33 & 105 & 23.63 & 24.81 & $1 \times 10^{-5}$ \\
L3 (CE) & 20.56 & 145 & 50.00 & 72.50 & $5 \times 10^{-5}$ \\
L4 (DE) & 25.06 & 145 & 61.73 & 89.51 & $8 \times 10^{-5}$ \\
\bottomrule
\end{tabular}
\caption{คุณสมบัติมวลของลิงก์ที่สกัดจากโมเดล CAD}
\end{table}

\textbf{หม้ายเหตุ:}
\begin{itemize}
    \item COM (Center of Mass) = ศูนย์กลางมวล
    \item $I_{zz}$ = โมเมนต์ความเฉื่อยรอบแกน z
    \item อัตราส่วน COM วัดจากข้อต่อเริ่มต้นของลิงก์
\end{itemize}

\section{การกำหนดวิถีการเดิน}

\subsection{วิถีรูปไข่ (Elliptical Gait Pattern)}

วิถีการเคลื่อนที่ของปลายเท้าเป็นรูปวงรีด้วยพารามิเตอร์ดังนี้:

\begin{align}
x_F(t) &= a \cos(\omega t) \\
y_F(t) &= y_{\text{home}} + b \sin(\omega t)
\end{align}

\textbf{พารามิเตอร์:}
\begin{itemize}
    \item แกนกึ่งยาว (Semi-major axis): $a = 60$ mm (ความยาวก้าว)
    \item แกนกึ่งสั้น (Semi-minor axis): $b = 30$ mm (ความสูงก้าว)
    \item ตำแหน่งยืน (Home position): $y_{\text{home}} = -200$ mm
    \item ความถี่เชิงมุม (Angular frequency): $\omega = 2\pi f = 2\pi$ rad/s
    \item ความถี่การเดิน (Gait frequency): $f = 1.0$ Hz
    \item เวลา 1 รอบ (Cycle time): $T = 1.0$ s
\end{itemize}

\textbf{หมายเหตุ:} อัตราส่วนแกนยาว:แกนสั้น = 2:1 เพื่อให้ได้วิถีรูปไข่ที่เหมาะสมกับการเดิน

\section{แบบจำลองไดนามิก}

\subsection{สมการการเคลื่อนที่}

สมการไดนามิกสำหรับระบบคือ:

\begin{equation}
\boldsymbol{\tau} = \mathbf{M}(\mathbf{q})\ddot{\mathbf{q}} + \mathbf{C}(\mathbf{q}, \dot{\mathbf{q}})\dot{\mathbf{q}} + \mathbf{G}(\mathbf{q})
\end{equation}

สำหรับการวิเคราะห์นี้ เราทำให้พจน์โคริโอลิสเป็นแบบง่าย จึงได้:

\begin{equation}
\boldsymbol{\tau} = \mathbf{M}(\mathbf{q})\ddot{\mathbf{q}} + \mathbf{G}(\mathbf{q})
\end{equation}

โดยที่:
\begin{itemize}
    \item $\boldsymbol{\tau} = [\tau_A, \tau_B]^T$ = ทอร์กที่ข้อต่อ
    \item $\mathbf{q} = [\theta_A, \theta_B]^T$ = มุมข้อต่อ
    \item $\mathbf{M}(\mathbf{q})$ = เมทริกซ์ความเฉื่อย (Inertia matrix) ขนาด $2 \times 2$
    \item $\mathbf{G}(\mathbf{q})$ = เวกเตอร์แรงโน้มถ่วง (Gravity vector) ขนาด $2 \times 1$
\end{itemize}

\subsection{เมทริกซ์ความเฉื่อย (Inertia Matrix)}

เมทริกซ์ความเฉื่อยคำนวณโดยใช้วิธี Composite Rigid Body:

\begin{equation}
\mathbf{M}(\mathbf{q}) = \sum_{i=1}^{4} \left[ m_i \mathbf{J}_{v,i}^T \mathbf{J}_{v,i} + I_{zz,i} \mathbf{J}_{\omega,i}^T \mathbf{J}_{\omega,i} \right]
\end{equation}

โดยที่:
\begin{itemize}
    \item $m_i$ = มวลของลิงก์ที่ $i$
    \item $\mathbf{J}_{v,i}$ = Jacobian ความเร็วเชิงเส้นของ COM ของลิงก์ที่ $i$ (ขนาด $2 \times 2$)
    \item $\mathbf{J}_{\omega,i}$ = Jacobian ความเร็วเชิงมุมของลิงก์ที่ $i$ (ขนาด $1 \times 2$)
    \item $I_{zz,i}$ = โมเมนต์ความเฉื่อยรอบแกน z
\end{itemize}

\subsection{เวกเตอร์แรงโน้มถ่วง (Gravity Vector)}

ทอร์กจากแรงโน้มถ่วงรวมส่วนสมทบจากทุกลิงก์และน้ำหนักหุ่นยนต์ที่กระจาย:

\begin{equation}
\mathbf{G}(\mathbf{q}) = \sum_{i=1}^{4} m_i g \mathbf{J}_{v,i}^T \begin{bmatrix} 0 \\ -1 \end{bmatrix} + m_{\text{leg}} g \mathbf{J}_F^T \begin{bmatrix} 0 \\ -1 \end{bmatrix}
\end{equation}

โดยที่:
\begin{itemize}
    \item $g = 9.81$ m/s² = ความเร่งโน้มถ่วง
    \item $m_{\text{leg}} = 1.675$ kg = มวลที่กระจายต่อขา
    \item $\mathbf{J}_F$ = Jacobian ของปลายเท้า (End-effector) ขนาด $2 \times 2$
\end{itemize}

\textbf{คำอธิบาย:} เวกเตอร์ $\begin{bmatrix} 0 \\ -1 \end{bmatrix}$ แสดงทิศของแรงโน้มถ่วงในแกน y (ชี้ลง)

\subsection{การคำนวณ Jacobian}

\subsubsection{ลิงก์ที่ 1 (AC) - ลิงก์มอเตอร์ซ้าย}

\begin{align}
\mathbf{J}_{v,1} &= \begin{bmatrix} -r_1 \sin\theta_A & 0 \\ r_1 \cos\theta_A & 0 \end{bmatrix} \\
\mathbf{J}_{\omega,1} &= \begin{bmatrix} 1 & 0 \end{bmatrix}
\end{align}

โดยที่ $r_1 = 0.03524$ m (ตำแหน่ง COM จากข้อต่อ A)

\subsubsection{ลิงก์ที่ 2 (BD) - ลิงก์มอเตอร์ขวา}

\begin{align}
\mathbf{J}_{v,2} &= \begin{bmatrix} 0 & -r_2 \sin\theta_B \\ 0 & r_2 \cos\theta_B \end{bmatrix} \\
\mathbf{J}_{\omega,2} &= \begin{bmatrix} 0 & 1 \end{bmatrix}
\end{align}

โดยที่ $r_2 = 0.02481$ m (ตำแหน่ง COM จากข้อต่อ B)

\subsubsection{ลิงก์ที่ 3 และ 4 (CE, DE) - ลิงก์เชื่อม}

สำหรับลิงก์เชื่อม Jacobian จะคำนวณด้วยวิธีเชิงตัวเลขตามท่าทางปัจจุบัน โดยคำนึงถึงจลนศาสตร์แบบเชื่อมโยงของกลไกแบบขนาน (Parallel Mechanism)

\section{การหาอนุพันธ์เชิงตัวเลข}

ความเร็วและความเร่งข้อต่อคำนวณโดยใช้วิธี Central Difference:

\subsection{ความเร็ว (Central Difference)}

\begin{equation}
\dot{\mathbf{q}}_i = \frac{\mathbf{q}_{i+1} - \mathbf{q}_{i-1}}{2\Delta t}
\end{equation}

\subsection{ความเร่ง (Central Difference)}

\begin{equation}
\ddot{\mathbf{q}}_i = \frac{\mathbf{q}_{i+1} - 2\mathbf{q}_i + \mathbf{q}_{i-1}}{(\Delta t)^2}
\end{equation}

โดยที่ $\Delta t = T/N = 1.0/60 \approx 0.0167$ s

\textbf{หมายเหตุ:} วิธี Central Difference ให้ความแม่นยำสูงกว่า Forward หรือ Backward Difference

\section{ผลการวิเคราะห์}

\subsection{ทอร์กไดนามิกสูงสุด}

การวิเคราะห์คำนวณทอร์กไดนามิกตลอด 1 รอบการเดิน (60 จุดข้อมูล) ค่าทอร์กสูงสุดคือ:

\begin{table}[H]
\centering
\begin{tabular}{@{}lcc@{}}
\toprule
\textbf{มอเตอร์} & \textbf{ทอร์กสูงสุด (N·m)} & \textbf{ตำแหน่งที่เกิด} \\ 
\midrule
มอเตอร์ A (ซ้าย) & 1.9273 & ระหว่างการเดิน \\
มอเตอร์ B (ขวา) & 1.6478 & ระหว่างการเดิน \\
\bottomrule
\end{tabular}
\caption{ความต้องการทอร์กไดนามิกสูงสุด (ผลจากการจำลอง)}
\end{table}

\subsection{เปรียบเทียบทอร์กแบบสถิตกับไดนามิก}

\begin{table}[H]
\centering
\begin{tabular}{@{}lccc@{}}
\toprule
\textbf{มอเตอร์} & \textbf{ทอร์กสถิต} & \textbf{ทอร์กไดนามิกสูงสุด} & \textbf{อัตราส่วน} \\ 
& (N·m) & (N·m) & (ไดนามิก/สถิต) \\
\midrule
มอเตอร์ A & 0.9172 & 1.9273 & 2.10× \\
มอเตอร์ B & 1.5628 & 1.6478 & 1.05× \\
\bottomrule
\end{tabular}
\caption{การเปรียบเทียบทอร์กแบบสถิตและไดนามิก}
\end{table}

\subsection{การตรวจสอบความเหมาะสมของมอเตอร์}

ข้อกำหนดมอเตอร์เป้าหมาย: \textbf{5.0 N·m}

\begin{table}[H]
\centering
\begin{tabular}{@{}lccc@{}}
\toprule
\textbf{มอเตอร์} & \textbf{ทอร์กสูงสุด} & \textbf{ค่า Safety Factor} & \textbf{สถานะ} \\ 
& (N·m) & & \\
\midrule
มอเตอร์ A & 1.9273 & 2.59× & \textcolor{green}{ผ่าน} \\
มอเตอร์ B & 1.6478 & 3.03× & \textcolor{green}{ผ่าน} \\
\bottomrule
\end{tabular}
\caption{การวิเคราะห์ค่า Safety Factor ของมอเตอร์}
\end{table}

\textbf{เกณฑ์การประเมิน Safety Factor:}
\begin{itemize}
    \item $SF \geq 2.0$ - ผ่าน (มอเตอร์เพียงพอ)
    \item $1.5 \leq SF < 2.0$ - เตือน (ค่า Safety Factor ต่ำ)
    \item $SF < 1.5$ - ไม่ผ่าน (มอเตอร์ไม่เพียงพอ)
\end{itemize}

\section{ข้อกำหนดมอเตอร์}

\subsection{พารามิเตอร์มอเตอร์ที่เลือก}

\begin{itemize}
    \item \textbf{ทอร์กขาออก:} 5.0 N·m (ข้อกำหนดเป้าหมาย)
    \item \textbf{อัตราทดเกียร์:} 8:1
    \item \textbf{ความเร็วขาออก:} 120 RPM (12.57 rad/s)
    \item \textbf{ความเร็วขาเข้า:} 960 RPM (ก่อนผ่านเกียร์บอกซ์)
\end{itemize}

\subsection{การตรวจสอบความเร็ว}

ความเร็วข้อต่อสูงสุดระหว่างรอบการเดินไม่ควรเกิน:

\begin{equation}
|\dot{\theta}_{\text{max}}| \leq 12.57 \text{ rad/s}
\end{equation}

\section{การวิเคราะห์องค์ประกอบของทอร์ก}

ทอร์กไดนามิกรวมสามารถแยกเป็นองค์ประกอบได้ดังนี้:

\subsection{ทอร์กเฉื่อย (Inertial Torque)}

\begin{equation}
\boldsymbol{\tau}_{\text{inertia}} = \mathbf{M}(\mathbf{q})\ddot{\mathbf{q}}
\end{equation}

คำนึงถึงความเร่งของลิงก์และโมเมนต์ความเฉื่อยในการหมุน

\subsection{ทอร์กโน้มถ่วง (Gravitational Torque)}

\begin{equation}
\boldsymbol{\tau}_{\text{gravity}} = \mathbf{G}(\mathbf{q})
\end{equation}

ประกอบด้วย:
\begin{itemize}
    \item น้ำหนักตัวลิงก์เอง (รวม 105.83 g)
    \item น้ำหนักหุ่นยนต์ที่กระจาย (1.675 kg ต่อขา)
\end{itemize}

\section{สรุปและข้อเสนอแนะ}

\subsection{ผลการศึกษาที่สำคัญ}

\begin{enumerate}
    \item คำนวณความต้องการทอร์กไดนามิกสำหรับการเดินแบบวิถีรูปไข่ที่ความถี่ 1 Hz
    \item กระจายน้ำหนักหุ่นยนต์ทั้งหมด (6.70 kg) อย่างถูกต้องทั้ง 4 ขา
    \item คำนวณค่า Safety Factor สำหรับมอเตอร์ขนาด 5.0 N·m
    \item ตรวจสอบขีดจำกัดความเร็วตามความสามารถของมอเตอร์
\end{enumerate}

\subsection{ข้อเสนอแนะ}

\begin{itemize}
    \item กรอกค่าทอร์กจริงจากการรันโปรแกรมจำลอง
    \item ตรวจสอบว่าค่า Safety Factor ผ่านเกณฑ์การออกแบบ ($SF \geq 2.0$)
    \item หากค่า Safety Factor ต่ำ ให้พิจารณา:
    \begin{itemize}
        \item ลดความถี่การเดิน
        \item ลดความยาวก้าว
        \item เลือกมอเตอร์ที่มีทอร์กสูงขึ้น
        \item ปรับวิถีการเดินให้ราบรื่นขึ้น
    \end{itemize}
    \item ดำเนินการต่อไปยัง Phase 3: การสร้างโมเดล URDF และการหาค่าที่เหมาะที่สุดของวิถี
\end{itemize}

\subsection{ขั้นตอนต่อไป}

\begin{enumerate}
    \item รันโปรแกรม \texttt{Dynamic-Torque-Analysis.py} เพื่อรับผลลัพธ์เชิงตัวเลข
    \item อัปเดตตารางด้วยค่าทอร์กสูงสุดที่ได้จริง
    \item ตรวจสอบความเหมาะสมของมอเตอร์กับเกณฑ์ความปลอดภัย
    \item สร้างกราฟแสดง:
    \begin{itemize}
        \item มุมข้อต่อตามเวลา
        \item ความเร็วและความเร่งข้อต่อ
        \item องค์ประกอบทอร์ก (เฉื่อยและโน้มถ่วง)
        \item ทอร์กไดนามิกรวมพร้อมขีดจำกัดมอเตอร์
    \end{itemize}
    \item บันทึกผลการศึกษาและดำเนินการเลือกฮาร์ดแวร์
\end{enumerate}

\section{เอกสารอ้างอิง}

\begin{enumerate}
    \item โมเดล CAD - คุณสมบัติวัสดุ PA12-HP Nylon
    \item Phase 2.1 - ผลการวิเคราะห์ทอร์กแบบสถิต (Static Torque Analysis)
    \item Craig, J.J., "Introduction to Robotics: Mechanics and Control"
    \item Merlet, J.P., "Parallel Robots"
\end{enumerate}

\appendix

\section{การใช้งานโปรแกรม Python}

การวิเคราะห์ทอร์กไดนามิกถูก implement ใน:

\begin{verbatim}
scripts/analysis/Dynamic-Torque-Analysis.py
\end{verbatim}

\textbf{ฟังก์ชันสำคัญ:}
\begin{itemize}
    \item \texttt{calculate\_inertia\_matrix(thetas)} - คำนวณ $\mathbf{M}(\mathbf{q})$
    \item \texttt{calculate\_gravity\_vector(thetas)} - คำนวณ $\mathbf{G}(\mathbf{q})$
    \item \texttt{calculate\_jacobian\_COM(thetas, link\_id)} - Jacobian ของ COM
    \item \texttt{calculate\_trajectory\_derivatives()} - การหาอนุพันธ์เชิงตัวเลข
\end{itemize}

\section{อัลกอริทึมการเลือก Configuration}

เพื่อให้การเคลื่อนที่ราบรื่น โปรแกรมแก้ inverse kinematics จะเลือก configuration ที่ทำให้ระยะทางเชิงมุมจากท่าก่อนหน้าน้อยที่สุด:

\begin{equation}
\text{config}^* = \arg\min_{\text{config}} \sum_{j=A,B} \min(|\theta_j - \theta_j^{\text{prev}}|, 2\pi - |\theta_j - \theta_j^{\text{prev}}|)
\end{equation}

วิธีนี้ป้องกันการกระโดดแบบไม่ต่อเนื่องระหว่าง configuration แบบ elbow-up และ elbow-down

\end{document}
