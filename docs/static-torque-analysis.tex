\documentclass[a4paper, 12pt]{article}

% --- PACKAGES ---
\usepackage{fontspec}
\usepackage{polyglossia}
\setmainlanguage{thai}
\setotherlanguage{english}
\defaultfontfeatures{Scale=MatchLowercase}
\setmainfont{TH SarabunPSK}
\newfontfamily\thaifont{TH SarabunPSK}
\usepackage{amsmath} % For math environments
\usepackage{geometry} % For page margins
\geometry{a4paper, margin=1in}
\usepackage{graphicx}
\usepackage{hyperref}
\usepackage{booktabs} % For professional tables
\usepackage{xcolor}

% --- TITLE ---
\title{การวิเคราะห์ทอร์กแบบสถิต \\ (Static Torque Analysis) \\ สำหรับกลไก 5-Bar Parallel Linkage}
\author{นายธีรโชติ เมืองจำนงค์}
\date{\today}

% --- DOCUMENT ---
\begin{document}

\maketitle

\section{บทนำ}

เอกสารนี้อธิบายการวิเคราะห์ทอร์ก (Torque) ที่มอเตอร์ทั้งสองต้องจ่ายเพื่อรองรับน้ำหนักของขาหุ่นยนต์ในท่านิ่ง (Static Condition) โดยใช้หลักการของ \textbf{Virtual Work} และ \textbf{Jacobian Transpose Method}

\textbf{วัตถุประสงค์:}
\begin{itemize}
    \item คำนวณทอร์กที่มอเตอร์ A และ B ต้องจ่ายในท่านิ่ง
    \item เปรียบเทียบทอร์กระหว่าง 3 Configurations ที่เป็นไปได้
    \item แนะนำ Configuration ที่เหมาะสมที่สุดสำหรับการใช้งาน
    \item เลือกฮาร์ดแวร์ (มอเตอร์) ที่เหมาะสม
\end{itemize}

\textbf{การประยุกต์ใช้:}
\begin{itemize}
    \item เลือกมอเตอร์ที่มีกำลังเพียงพอ
    \item ประเมินประสิทธิภาพของแต่ละ Configuration
    \item วางแผนการควบคุมให้ใช้พลังงานต่ำสุด
\end{itemize}

\section{ทฤษฎี: Jacobian Transpose Method}

\subsection{หลักการ Virtual Work}

ในท่านิ่ง (Static Equilibrium) งานเสมือน (Virtual Work) ที่ทำโดยแรงภายนอก $F$ และทอร์กภายใน $\tau$ ต้องเท่ากัน:

\begin{equation}
\delta W_{external} = \delta W_{internal}
\end{equation}

\begin{equation}
F^T \delta P_F = \tau^T \delta q
\end{equation}

โดย:
\begin{itemize}
    \item $F = [F_x, F_y]^T$ = แรงที่กระทำต่อปลายเท้า (End-effector) ที่จุด $F$
    \item $\delta P_F = [\delta x, \delta y]^T$ = การเคลื่อนที่เสมือนของปลายเท้า
    \item $\tau = [\tau_A, \tau_B]^T$ = ทอร์กที่มอเตอร์ A และ B
    \item $\delta q = [\delta\theta_A, \delta\theta_B]^T$ = การหมุนเสมือนของมอเตอร์
\end{itemize}

\subsection{ความสัมพันธ์ผ่าน Jacobian}

จากการวิเคราะห์จลนศาสตร์เชิงขับ (Forward Kinematics) เรารู้ว่า:

\begin{equation}
\delta P_F = J_F \delta q
\end{equation}

โดย $J_F$ คือ \textbf{Jacobian Matrix} ขนาด $2 \times 2$ ที่ได้จากการอนุพันธ์ของ Forward Kinematics:

\begin{equation}
J_F = \begin{bmatrix}
\frac{\partial x_F}{\partial \theta_A} & \frac{\partial x_F}{\partial \theta_B} \\
\frac{\partial y_F}{\partial \theta_A} & \frac{\partial y_F}{\partial \theta_B}
\end{bmatrix}
\end{equation}

(ดูรายละเอียดการ derive ใน \texttt{forward-kinematics-5bar.tex} Section 3)

\subsection{สูตรทอร์กแบบสถิต}

แทนค่า $\delta P_F = J_F \delta q$ ลงในสมการ Virtual Work:

\begin{equation}
F^T (J_F \delta q) = \tau^T \delta q
\end{equation}

\begin{equation}
(J_F^T F)^T \delta q = \tau^T \delta q
\end{equation}

เนื่องจากสมการต้องเป็นจริงสำหรับทุก $\delta q$ ดังนั้น:

\begin{equation}
\boxed{\tau = J_F^T F}
\end{equation}

นี่คือ \textbf{Jacobian Transpose Method} ที่ใช้คำนวณทอร์กแบบสถิต

\section{พารามิเตอร์และข้อมูลเบื้องต้น}

\subsection{พารามิเตอร์ทางกายภาพ}

\begin{table}[h]
\centering
\begin{tabular}{lcc}
\toprule
\textbf{พารามิเตอร์} & \textbf{ค่า} & \textbf{หน่วย} \\
\midrule
มวลรวมของขา & 7.22 & kg \\
แรงโน้มถ่วง $g$ & 9.81 & m/s$^2$ \\
แรงโน้มถ่วงรวม $mg$ & 17.71 & N \\
\bottomrule
\end{tabular}
\caption{พารามิเตอร์ทางกายภาพของขาหุ่นยนต์}
\end{table}

\subsection{แรงที่กระทำต่อปลายเท้า}

ในท่านิ่ง แรงที่กระทำต่อปลายเท้าคือแรงโน้มถ่วงเท่านั้น (ไม่มีแรงภายนอกอื่น):

\begin{equation}
F = \begin{bmatrix}
F_x \\
F_y
\end{bmatrix} = \begin{bmatrix}
0 \\
-mg
\end{bmatrix} = \begin{bmatrix}
0 \\
-17.71
\end{bmatrix} \text{ N}
\end{equation}

เครื่องหมาย $-$ บ่งชี้ว่าแรงชี้ลงตามแกน $y$

\subsection{Home Pose และ Configurations}

สำหรับท่าอ้างอิง (Home Pose) ที่ปลายเท้าอยู่ที่:

\begin{equation}
P_F = (0, -200) \text{ mm}
\end{equation}

จากการวิเคราะห์ Inverse Kinematics (ดู \texttt{inverse-kinematics-analytical.tex}) พบว่ามี \textbf{3 Configurations ที่เป็นไปได้} (Valid Solutions):

\begin{table}[h]
\centering
\begin{tabular}{lccc}
\toprule
\textbf{Configuration} & \textbf{$\theta_A$ (°)} & \textbf{$\theta_B$ (°)} & \textbf{สถานะ} \\
\midrule
Config 1 (Down-Down) & -119.53 & -37.68 & \color{green}Valid \\
Config 2 (Down-Up) & -139.91 & -166.32 & \color{green}Valid \\
Config 3 (Up-Down) & -16.06 & -37.68 & \color{green}Valid \\
Config 4 (Up-Up) & - & - & \color{red}Invalid \\
\bottomrule
\end{tabular}
\caption{ทั้งหมด 4 Configurations สำหรับ Home Pose (มี 3 แบบที่ใช้งานได้)}
\end{table}

\section{การคำนวณทอร์กสำหรับแต่ละ Configuration}

\subsection{ขั้นตอนการคำนวณ}

สำหรับแต่ละ Configuration เราจะคำนวณทอร์กดังนี้:

\textbf{ขั้นตอนที่ 1:} คำนวณ Jacobian Matrix $J_F$ ที่ Configuration นั้นๆ โดยใช้ฟังก์ชัน:
\begin{verbatim}
J_F = calculate_jacobian([theta_A, theta_B])
\end{verbatim}

\textbf{ขั้นตอนที่ 2:} คำนวณทอร์กโดยใช้สูตร:
\begin{equation}
\tau = J_F^T F = J_F^T \begin{bmatrix} 0 \\ -17.71 \end{bmatrix}
\end{equation}

\textbf{ขั้นตอนที่ 3:} แยกองค์ประกอบของทอร์ก:
\begin{equation}
\tau = \begin{bmatrix}
\tau_A \\
\tau_B
\end{bmatrix}
\end{equation}

\subsection{Configuration 1: Down-Down (แนะนำ)}

\textbf{มุมมอเตอร์:}
\begin{equation}
\theta_A = -119.53°, \quad \theta_B = -37.68°
\end{equation}

\textbf{Jacobian Matrix:} (คำนวณจากฟังก์ชัน \texttt{calculate\_jacobian})
\begin{equation}
J_F = \begin{bmatrix}
J_{11} & J_{12} \\
J_{21} & J_{22}
\end{bmatrix}
\end{equation}

\textbf{ทอร์ก:}
\begin{equation}
\tau = J_F^T \begin{bmatrix} 0 \\ -17.71 \end{bmatrix} = \begin{bmatrix}
J_{21} \times (-17.71) \\
J_{22} \times (-17.71)
\end{bmatrix}
\end{equation}

\textbf{ผลลัพธ์โดยประมาณ:}
\begin{equation}
\tau_A \approx 1.63 \text{ N-m}, \quad \tau_B \approx 1.60 \text{ N-m}
\end{equation}

\textbf{คุณสมบัติ:}
\begin{itemize}
    \item[\color{green}$+$] ทอร์กสมดุล: $|\tau_A| \approx |\tau_B|$ (แตกต่างกันเพียง 2\%)
    \item[\color{green}$+$] มุมทั้งสองอยู่ในช่วงปลอดภัย (ห่างจากขีดจำกัด)
    \item[\color{green}$+$] Elbow Down ทั้งสองข้าง (ธรรมชาติ สำหรับการยืน)
\end{itemize}

\subsection{Configuration 2: Down-Up}

\textbf{มุมมอเตอร์:}
\begin{equation}
\theta_A = -139.91°, \quad \theta_B = -166.32°
\end{equation}

\textbf{ผลลัพธ์โดยประมาณ:}
\begin{equation}
\tau_A \approx 2.1 \text{ N-m}, \quad \tau_B \approx 3.5 \text{ N-m}
\end{equation}

\textbf{คุณสมบัติ:}
\begin{itemize}
    \item[\color{orange}$\sim$] ทอร์กไม่สมดุล: $\tau_B$ สูงกว่า $\tau_A$ มาก (67\%)
    \item[\color{red}$-$] มุม $\theta_B = -166.32°$ ใกล้ขีดจำกัด -180°
    \item[\color{orange}$\sim$] Configuration แปลกและไม่ธรรมชาติ
\end{itemize}

\subsection{Configuration 3: Up-Down}

\textbf{มุมมอเตอร์:}
\begin{equation}
\theta_A = -16.06°, \quad \theta_B = -37.68°
\end{equation}

\textbf{ผลลัพธ์โดยประมาณ:}
\begin{equation}
\tau_A \approx 0.8 \text{ N-m}, \quad \tau_B \approx 1.6 \text{ N-m}
\end{equation}

\textbf{คุณสมบัติ:}
\begin{itemize}
    \item[\color{orange}$\sim$] ทอร์กไม่สมดุล: $\tau_B$ สูงกว่า $\tau_A$ (100\%)
    \item[\color{green}$+$] มุมอยู่ในช่วงปลอดภัย
    \item[\color{red}$-$] Elbow Up ด้านซ้าย (ไม่เหมาะสมสำหรับการยืน)
\end{itemize}

\section{การเปรียบเทียบและข้อแนะนำ}

\subsection{ตารางเปรียบเทียบ}

\begin{table}[h]
\centering
\begin{tabular}{lcccc}
\toprule
\textbf{Configuration} & \textbf{$\tau_A$ (N-m)} & \textbf{$\tau_B$ (N-m)} & \textbf{ความสมดุล} & \textbf{ข้อเสนอแนะ} \\
\midrule
Config 1 (Down-Down) & 1.63 & 1.60 & \color{green}ดีเยี่ยม (2\%) & \color{green}⭐ แนะนำ \\
Config 2 (Down-Up) & 2.1 & 3.5 & \color{red}แย่ (67\%) & \color{red}ไม่แนะนำ \\
Config 3 (Up-Down) & 0.8 & 1.6 & \color{orange}ปานกลาง (100\%) & \color{orange}ใช้ได้ \\
\bottomrule
\end{tabular}
\caption{เปรียบเทียบทอร์กของทั้ง 3 Configurations}
\end{table}

\textit{หมายเหตุ:} ความสมดุล = $\frac{|\tau_B - \tau_A|}{\tau_A} \times 100\%$

\subsection{ข้อแนะนำ}

\textbf{Configuration ที่แนะนำ:} \textcolor{green}{\textbf{Config 1 (Down-Down)}}

\textbf{เหตุผล:}
\begin{enumerate}
    \item \textbf{ทอร์กสมดุล:} มอเตอร์ทั้งสองใช้กำลังใกล้เคียงกัน ลดการสึกหรอไม่สม่ำเสมอ
    \item \textbf{ประสิทธิภาพพลังงาน:} ใช้พลังงานรวมต่ำสุด
    \item \textbf{ความปลอดภัย:} มุมทั้งสองห่างจากขีดจำกัด
    \item \textbf{ธรรมชาติ:} Elbow Down เหมาะสมสำหรับการยืนและเดิน
\end{enumerate}

\section{การเลือกมอเตอร์}

\subsection{ทอร์กสูงสุดที่ต้องการ}

จากการวิเคราะห์ Static Torque พบว่า:

\begin{itemize}
    \item ทอร์กสูงสุดในท่านิ่ง: $\tau_{max} \approx 1.63$ N-m
    \item เพิ่ม Safety Factor 3× สำหรับการเคลื่อนที่: $\tau_{required} = 1.63 \times 3 \approx 5$ N-m
\end{itemize}

\subsection{มอเตอร์ที่เลือก}

\textbf{ข้อเสนอแนะ:} เลือกสเต็ปมอเตอร์ที่มี \textbf{Holding Torque $\geq$ 5 N-m}

\textbf{ตัวอย่าง:}
\begin{itemize}
    \item NEMA 23 (57 mm) - 5 N-m Stepper Motor
    \item มี Safety Factor เพียงพอสำหรับการเดินและการเร่งความเร็ว
\end{itemize}

\subsection{ข้อควรพิจารณาเพิ่มเติม}

\begin{itemize}
    \item \textbf{Dynamic Torque:} การวิเคราะห์ในเอกสารนี้เป็นแบบสถิตเท่านั้น สำหรับการเคลื่อนที่จริง ต้องคำนวณ Dynamic Torque ที่รวม Inertia และ Acceleration ด้วย (Phase 2.2)
    \item \textbf{Peak Torque:} ในระหว่างการเดิน ทอร์กสูงสุดอาจมากกว่า Static Torque 2-3 เท่า
    \item \textbf{Continuous vs Peak:} ตรวจสอบว่ามอเตอร์สามารถจ่ายทอร์กสูงสุดได้นานเท่าใด
\end{itemize}

\section{สรุป}

\subsection{ผลการวิเคราะห์}

\begin{enumerate}
    \item ได้สูตรคำนวณทอร์กแบบสถิต: $\tau = J_F^T F$
    \item คำนวณทอร์กสำหรับ 3 Valid Configurations ที่ Home Pose
    \item \textbf{Config 1 (Down-Down)} ให้ผลลัพธ์ดีที่สุด (ทอร์กสมดุล)
    \item ทอร์กสูงสุด $\approx$ 1.63 N-m (ท่านิ่ง)
    \item แนะนำเลือกมอเตอร์ 5 N-m (รวม Safety Factor 3×)
\end{enumerate}

\subsection{งานต่อไป (Phase 2.2)}

\begin{itemize}
    \item คำนวณ \textbf{Inertia Matrix} $M(q)$ จากมวลและ COM ของแต่ละ link
    \item คำนวณ \textbf{Gravity Vector} $G(q)$
    \item คำนวณ \textbf{Dynamic Torque}: $\tau = M(q)\ddot{q} + C(q,\dot{q})\dot{q} + G(q) - J_F^T F_{ext}$
    \item จำลอง Trajectory และหา \textbf{Peak Torque} ในระหว่างการเดิน
\end{itemize}

\section{อ้างอิง}

\begin{itemize}
    \item \texttt{forward-kinematics-5bar.tex} - Forward Kinematics และ Jacobian Derivation
    \item \texttt{inverse-kinematics-analytical.tex} - Inverse Kinematics (4 Configurations)
    \item \texttt{IK-Five-Bar-Leg-Analytical.py} - Python Implementation
\end{itemize}

\end{document}
