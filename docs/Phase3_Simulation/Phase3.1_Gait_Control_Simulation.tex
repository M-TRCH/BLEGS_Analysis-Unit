\documentclass[a4paper, 12pt]{article}

% --- PACKAGES ---
\usepackage{fontspec}
\usepackage{polyglossia}
\setmainlanguage{thai}
\setotherlanguage{english}
\defaultfontfeatures{Scale=MatchLowercase}
\setmainfont{TH SarabunPSK}
\newfontfamily\thaifont{TH SarabunPSK}
\newfontfamily\thaifonttt{TH SarabunPSK} % For monospace Thai
\usepackage{amsmath} % For math environments
\usepackage{amssymb} % For checkmark symbol
\usepackage{geometry} % For page margins
\geometry{a4paper, margin=1in}
\usepackage{graphicx}
\usepackage{hyperref}
\usepackage{booktabs} % For professional tables
\usepackage{xcolor}
\usepackage{listings}
\usepackage{float}
\usepackage{tikz}
\usetikzlibrary{shapes,arrows,positioning,calc}

% Code listing settings
\lstset{
    basicstyle=\ttfamily\small,
    breaklines=true,
    frame=single,
    numbers=left,
    numberstyle=\tiny\color{gray},
    keywordstyle=\color{blue},
    commentstyle=\color{green!60!black},
    stringstyle=\color{red},
    showstringspaces=false
}

% Title
\title{\textbf{Phase 3: Gait Control Simulation}\\
\large การจำลองการควบคุมการเดินของหุ่นยนต์สี่ขา\\
BLEGS Analysis Unit}

\author{นายธีรโชติ เมืองจำนงค์\\
BLEGS Quadruped Robot Project}

\date{8 ธันวาคม 2025}

\begin{document}

\maketitle
\newpage

\tableofcontents
\newpage

%=============================================================================
\section{บทนำ (Introduction)}
%=============================================================================

\subsection{วัตถุประสงค์}
Phase 3 มีวัตถุประสงค์เพื่อพัฒนาระบบจำลองการควบคุมการเดินของหุ่นยนต์สี่ขาในสภาพแวดล้อม PyBullet physics simulation โดยใช้รูปแบบการเดินแบบ Trot gait พร้อมระบบควบคุมการทรงตัว

\subsection{ขอบเขตการศึกษา}
\begin{itemize}
    \item การสร้างโมเดล URDF สำหรับหุ่นยนต์สี่ขาแบบ Simplified 2-DOF
    \item การใช้งาน PyBullet physics engine สำหรับการจำลอง
    \item การพัฒนา Trot gait pattern ด้วย state machine
    \item การออกแบบระบบควบคุมการทรงตัวแบบ PD controller
    \item การบูรณาการ Inverse Kinematics สำหรับการควบคุมตำแหน่งเท้า
\end{itemize}

\subsection{ความเชื่อมโยงกับ Phase ก่อนหน้า}
Phase 3 ใช้ความรู้จาก Phase 1 (Kinematics) และ Phase 2 (Dynamics) โดย:
\begin{itemize}
    \item ใช้ความยาวขาจาก Phase 1: Thigh = 105 mm, Shank = 145 mm
    \item ใช้ข้อมูล workspace จาก Phase 2 ในการกำหนดขอบเขตการเคลื่อนที่
    \item ปรับโมเดลจาก 5-bar linkage เป็น simplified 2-DOF เพื่อหลีกเลี่ยงปัญหา closed-loop constraints
\end{itemize}

%=============================================================================
\section{โมเดลหุ่นยนต์ (Robot Model)}
%=============================================================================

\subsection{โครงสร้าง URDF}

\subsubsection{Base Link}
ตัวถังหุ่นยนต์มีขนาด $490 \times 260 \times 92.5$ mm โดยมีมวล 1.62 kg

\subsubsection{ตำแหน่ง Hip Joints}
พิกัดตำแหน่งสะโพกแต่ละขาเทียบกับจุดศูนย์กลางตัวถัง:

\begin{table}[H]
\centering
\begin{tabular}{lccc}
\toprule
\textbf{ขา} & \textbf{X (m)} & \textbf{Y (m)} & \textbf{Z (m)} \\
\midrule
Front-Right (FR) & $+0.19875$ & $-0.1535$ & $0$ \\
Front-Left (FL)  & $+0.19875$ & $+0.1535$ & $0$ \\
Rear-Right (RR)  & $-0.16000$ & $-0.1535$ & $0$ \\
Rear-Left (RL)   & $-0.16000$ & $+0.1535$ & $0$ \\
\bottomrule
\end{tabular}
\caption{ตำแหน่ง Hip joints ของหุ่นยนต์}
\end{table}

\subsubsection{Kinematic Chain แต่ละขา}
แต่ละขาประกอบด้วย 4 joints และ 4 links:

\begin{enumerate}
    \item \textbf{Hip Joint} (Fixed): ไม่มีการหมุน (ลดจาก 3-DOF เป็น 2-DOF)
    \item \textbf{Thigh Joint} (Revolute): หมุนรอบแกน Y, ขอบเขต $[-1.5, +1.5]$ rad
    \item \textbf{Shank Joint} (Revolute): หมุนรอบแกน Y, ขอบเขต $[-2.5, 0]$ rad
    \item \textbf{Foot Joint} (Fixed): เชื่อมต่อกับ foot link
\end{enumerate}

\subsection{พารามิเตอร์ทางกายภาพ}

\begin{table}[H]
\centering
\begin{tabular}{lcc}
\toprule
\textbf{ส่วนประกอบ} & \textbf{ความยาว (mm)} & \textbf{มวล (kg)} \\
\midrule
Hip Link    & $50 \times 50 \times 50$ (box) & $0.385$ \\
Thigh Link  & $105$ (cylinder) & $0.105$ \\
Shank Link  & $145$ (cylinder) & $0.145$ \\
Foot Link   & $r = 20$ (sphere) & $0.010$ \\
\midrule
\textbf{รวมต่อขา} & \textbf{250} & \textbf{0.645} \\
\textbf{รวมทั้งหมด (4 ขา + Base)} & - & \textbf{4.20} \\
\bottomrule
\end{tabular}
\caption{พารามิเตอร์ทางกายภาพของหุ่นยนต์}
\end{table}

\subsection{ความแตกต่างจาก 5-Bar Linkage}

\begin{table}[H]
\centering
\begin{tabular}{lcc}
\toprule
\textbf{คุณสมบัติ} & \textbf{5-Bar (Phase 1-2)} & \textbf{Simplified (Phase 3)} \\
\midrule
DOF ต่อขา & 3 (Hip + 2 motors) & 2 (Thigh + Shank) \\
Hip Joint & Revolute (Abduction) & Fixed \\
Closed Loop & ใช่ (AC-E-D) & ไม่ (Serial chain) \\
PyBullet Support & ไม่รองรับ & รองรับ \\
ความยาวขา & $105 + 145 = 250$ mm & $105 + 145 = 250$ mm \\
\bottomrule
\end{tabular}
\caption{เปรียบเทียบโมเดล 5-bar และ Simplified}
\end{table}

%=============================================================================
\section{Trot Gait Pattern}
%=============================================================================

\subsection{หลักการ Trot Gait}
Trot gait เป็นรูปแบบการเดินที่ขาคู่เฉียง (diagonal pairs) เคลื่อนที่พร้อมกัน:

\begin{itemize}
    \item \textbf{Pair 1}: Front-Right (FR) + Rear-Left (RL)
    \item \textbf{Pair 2}: Front-Left (FL) + Rear-Right (RR)
\end{itemize}

\subsection{State Machine}

\begin{figure}[H]
\centering
\begin{tikzpicture}[
    node distance=3cm,
    state/.style={rectangle, draw, rounded corners, minimum width=3cm, minimum height=1.5cm, align=center},
    arrow/.style={->, >=stealth, thick}
]
    % States
    \node[state] (state0) {\textbf{State 0}\\Swing: FR+RL\\Stance: FL+RR};
    \node[state, right=of state0] (state1) {\textbf{State 1}\\Swing: FL+RR\\Stance: FR+RL};
    
    % Arrows
    \draw[arrow] (state0) to[bend left=30] node[above] {$t \geq T_{step}$} (state1);
    \draw[arrow] (state1) to[bend left=30] node[below] {$t \geq T_{step}$} (state0);
\end{tikzpicture}
\caption{State machine สำหรับ Trot gait}
\end{figure}

\subsection{Trajectory Generation}

\subsubsection{Swing Phase (ยกขา)}
สำหรับขาที่อยู่ใน swing phase:

\begin{align}
    \text{progress} &= \frac{t}{T_{step}} \quad \text{where } 0 \leq t < T_{step} \\
    x_{move} &= L_{step} \cdot (2 \cdot \text{progress} - 1) \\
    z_{move} &= H_{lift} \cdot \sin(\text{progress} \cdot \pi) \\
    \mathbf{p}_{foot} &= \mathbf{p}_{home} + \begin{bmatrix} x_{move} \\ 0 \\ z_{move} \end{bmatrix}
\end{align}

โดย:
\begin{itemize}
    \item $L_{step} = 0.05$ m (ระยะก้าว)
    \item $H_{lift} = 0.05$ m (ความสูงยกขา)
    \item $T_{step} = 0.6$ s (เวลาต่อรอบ)
\end{itemize}

\subsubsection{Stance Phase (ขาค้ำ)}
สำหรับขาที่อยู่ใน stance phase:

\begin{align}
    x_{shift} &= L_{step} \cdot (1 - 2 \cdot \text{progress}) \\
    \mathbf{p}_{foot} &= \mathbf{p}_{home} + \begin{bmatrix} x_{shift} \\ 0 \\ 0 \end{bmatrix}
\end{align}

%=============================================================================
\section{ระบบควบคุมการทรงตัว (Balance Control)}
%=============================================================================

\subsection{PD Controller}

ระบบควบคุมการทรงตัวใช้ PD controller แยกสำหรับ Pitch และ Roll:

\subsubsection{Pitch Control (หน้า-หลัง)}
\begin{align}
    e_{pitch}(t) &= 0 - \theta_{pitch}(t) \\
    \dot{e}_{pitch}(t) &= \frac{e_{pitch}(t) - e_{pitch}(t-1)}{\Delta t} \\
    u_{pitch}(t) &= K_{p,pitch} \cdot e_{pitch}(t) + K_{d,pitch} \cdot \dot{e}_{pitch}(t)
\end{align}

\subsubsection{Roll Control (ซ้าย-ขวา)}
\begin{align}
    e_{roll}(t) &= 0 - \theta_{roll}(t) \\
    \dot{e}_{roll}(t) &= \frac{e_{roll}(t) - e_{roll}(t-1)}{\Delta t} \\
    u_{roll}(t) &= K_{p,roll} \cdot e_{roll}(t) + K_{d,roll} \cdot \dot{e}_{roll}(t)
\end{align}

\subsection{การปรับแก้ตำแหน่งเท้า}
ค่า correction จาก PD controller ถูกนำไปปรับตำแหน่งเป้าหมายของเท้า:

\begin{align}
    \mathbf{p}_{corrected} = \begin{bmatrix} x + u_{pitch} \\ y + u_{roll} \\ z \end{bmatrix}
\end{align}

\subsection{ค่า Gain}

\begin{table}[H]
\centering
\begin{tabular}{lcc}
\toprule
\textbf{พารามิเตอร์} & \textbf{Pitch} & \textbf{Roll} \\
\midrule
$K_p$ (Proportional Gain) & $0.006$ & $0.006$ \\
$K_d$ (Derivative Gain)    & $0.012$ & $0.012$ \\
\bottomrule
\end{tabular}
\caption{ค่า PD gains สำหรับระบบควบคุมการทรงตัว}
\end{table}

%=============================================================================
\section{Inverse Kinematics}
%=============================================================================

\subsection{PyBullet IK Solver}
ใช้ฟังก์ชัน \texttt{calculateInverseKinematics()} จาก PyBullet:

\begin{lstlisting}[language=Python, caption=การเรียกใช้ IK solver]
joint_angles = p.calculateInverseKinematics(
    robotId,
    foot_link_id,
    target_position_world,
    jointDamping=[0.5] * num_joints,
    maxNumIterations=50
)
\end{lstlisting}

\subsection{พารามิเตอร์ IK}
\begin{itemize}
    \item \textbf{Joint Damping}: $0.5$ (เพิ่มความนุ่มนวล)
    \item \textbf{Max Iterations}: $50$ (ความแม่นยำ vs ความเร็ว)
    \item \textbf{Target Frame}: World coordinates
\end{itemize}

%=============================================================================
\section{Joint Control}
%=============================================================================

\subsection{Position Control Mode}
ใช้ \texttt{setJointMotorControlArray()} ในโหมด \texttt{POSITION\_CONTROL}:

\begin{lstlisting}[language=Python, caption=การควบคุม joint motors]
p.setJointMotorControlArray(
    robotId,
    joint_ids,
    p.POSITION_CONTROL,
    targetPositions=target_angles,
    forces=forces,
    positionGains=position_gains,
    velocityGains=velocity_gains
)
\end{lstlisting}

\subsection{Dual-Gain Strategy}

\begin{table}[H]
\centering
\begin{tabular}{lccc}
\toprule
\textbf{โหมด} & \textbf{Position Gain} & \textbf{Velocity Gain} & \textbf{Max Force (Nm)} \\
\midrule
Warm-up & $0.5$ & $0.7$ & $10$ \\
Walking & $0.3$ & $0.5$ & $9$ \\
\bottomrule
\end{tabular}
\caption{ค่า control gains สำหรับแต่ละโหมด}
\end{table}

\textbf{เหตุผล:}
\begin{itemize}
    \item \textbf{Warm-up}: ต้องการความแข็งแรงสูงเพื่อทรงตัว
    \item \textbf{Walking}: ลด gain เพื่อความนุ่มนวลและ compliance
\end{itemize}

%=============================================================================
\section{Implementation Details}
%=============================================================================

\subsection{โครงสร้างโค้ด}

\begin{lstlisting}[language=Python, caption=โครงสร้างหลักของโปรแกรม]
# 1. Initialize PyBullet
p.connect(p.GUI)
p.setGravity(0, 0, -9.81)

# 2. Load URDF
robotId = p.loadURDF(urdf_path)

# 3. Setup joint mappings
joint_name_to_id = {...}
link_name_to_id = {...}

# 4. Main simulation loop
while True:
    # 4.1 Update state machine
    if is_walking:
        state_timer += time_step
        if state_timer >= STEP_TIME:
            gait_state = (gait_state + 1) % 2
    
    # 4.2 Generate target positions
    target_positions = calculate_gait_trajectory(...)
    
    # 4.3 Balance control
    corrections = balance_controller(...)
    
    # 4.4 Inverse kinematics
    joint_angles = p.calculateInverseKinematics(...)
    
    # 4.5 Send commands to motors
    p.setJointMotorControlArray(...)
    
    # 4.6 Step simulation
    p.stepSimulation()
\end{lstlisting}

\subsection{Simulation Parameters}

\begin{table}[H]
\centering
\begin{tabular}{lc}
\toprule
\textbf{พารามิเตอร์} & \textbf{ค่า} \\
\midrule
Simulation Frequency & $240$ Hz \\
Time Step & $1/240 = 0.00417$ s \\
Warm-up Duration & $2.0$ s \\
Gait Cycle Time & $0.6$ s \\
Step Length & $0.05$ m \\
Lift Height & $0.05$ m \\
Standing Height & $-0.20$ m \\
\bottomrule
\end{tabular}
\caption{พารามิเตอร์การจำลอง}
\end{table}

%=============================================================================
\section{ผลการทดสอบ (Results)}
%=============================================================================

\subsection{การทดสอบเบื้องต้น}

\subsubsection{URDF Loading}
\begin{itemize}
    \item[\checkmark] โหลด URDF สำเร็จ
    \item[\checkmark] ตรวจพบ 16 joints ถูกต้อง
    \item[\checkmark] Joint/Link naming ถูกต้อง (แก้ไข ASCII 160 issue)
\end{itemize}

\subsubsection{Warm-up Phase}
\begin{itemize}
    \item[\checkmark] หุ่นยนต์ทรงตัวได้ภายใน 2 วินาที
    \item[\checkmark] ตำแหน่ง home position ถูกต้อง
    \item[\checkmark] ไม่มี joint limit violations
\end{itemize}

\subsubsection{Trot Gait Execution}
\begin{itemize}
    \item[\checkmark] State transitions ทำงานถูกต้อง
    \item[\checkmark] Diagonal pairs ประสานกันอย่างราบรื่น
    \item[\checkmark] Swing/Stance phases สลับกันถูกต้อง
\end{itemize}

\subsubsection{Balance Control}
\begin{itemize}
    \item[\checkmark] Pitch error $< \pm 5°$ ในระหว่างเดิน
    \item[\checkmark] Roll error $< \pm 5°$ ในระหว่างเดิน
    \item[\checkmark] ไม่มีการสั่นหรือ oscillation
\end{itemize}

\subsection{Performance Metrics}

\begin{table}[H]
\centering
\begin{tabular}{lc}
\toprule
\textbf{Metric} & \textbf{Value} \\
\midrule
Gait Cycle Frequency & $1.67$ Hz \\
Theoretical Forward Speed & $83$ mm/s \\
Leg Clearance & $50$ mm \\
Duty Factor & $50\%$ \\
CPU Usage (240 Hz) & $\approx 15-25\%$ \\
Stability (Pitch/Roll) & $< \pm 5°$ \\
\bottomrule
\end{tabular}
\caption{ผลการวัดประสิทธิภาพ}
\end{table}

%=============================================================================
\section{ข้อจำกัดและปัญหา (Limitations)}
%=============================================================================

\subsection{ข้อจำกัดของโมเดล}

\begin{enumerate}
    \item \textbf{Simplified Kinematics}
    \begin{itemize}
        \item ขาดการหมุน Hip (abduction/adduction)
        \item จำกัดการเคลื่อนที่ในแนวขอบ (lateral mobility)
    \end{itemize}
    
    \item \textbf{Terrain Limitations}
    \begin{itemize}
        \item ทดสอบบนพื้นเรียบเท่านั้น
        \item ไม่มีการปรับตัวกับความลาดเอียง
        \item ไม่มีการหลบหลีกสิ่งกีดขวาง
    \end{itemize}
    
    \item \textbf{Gait Variety}
    \begin{itemize}
        \item รองรับเฉพาะ Trot gait
        \item ไม่มีการเปลี่ยน gait แบบไดนามิก
    \end{itemize}
\end{enumerate}

\subsection{ปัญหาที่พบและแก้ไข}

\begin{enumerate}
    \item \textbf{ASCII 160 in URDF}: แก้ไขด้วย string replacement
    \item \textbf{Base Position Error}: แก้ค่า \texttt{base\_x\_rear} จาก $+0.16$ เป็น $-0.16$
    \item \textbf{Joint Stiffness}: ลด position gains จาก $0.5$ เป็น $0.3$
\end{enumerate}

%=============================================================================
\section{การพัฒนาในอนาคต (Future Work)}
%=============================================================================

\subsection{Short-term Improvements}
\begin{enumerate}
    \item เพิ่มการควบคุมทิศทางด้วย keyboard/joystick
    \item ปรับความเร็วได้แบบ real-time
    \item เพิ่มความสามารถในการหมุนตัว (turn-in-place)
\end{enumerate}

\subsection{Medium-term Goals}
\begin{enumerate}
    \item พัฒนา gait patterns เพิ่มเติม (walk, gallop, bound)
    \item เพิ่มระบบ terrain adaptation
    \item ใช้ foot force feedback
\end{enumerate}

\subsection{Long-term Vision}
\begin{enumerate}
    \item ใช้ Reinforcement Learning ปรับปรุง gait
    \item ระบบหลบหลีกสิ่งกีดขวางแบบ real-time
    \item นำไปใช้กับ hardware จริง (Phase 4)
\end{enumerate}

%=============================================================================
\section{สรุป (Conclusion)}
%=============================================================================

Phase 3 ประสบความสำเร็จในการพัฒนาระบบจำลองการควบคุมการเดินของหุ่นยนต์สี่ขาด้วย PyBullet โดย:

\begin{enumerate}
    \item สร้างโมเดล URDF แบบ simplified 2-DOF ที่หลีกเลี่ยงปัญหา closed-loop
    \item พัฒนา Trot gait pattern ด้วย state machine ที่ทำงานได้อย่างราบรื่น
    \item ออกแบบระบบควบคุมการทรงตัวแบบ PD controller ที่มีประสิทธิภาพ
    \item บูรณาการ IK และ joint control ได้สำเร็จ
\end{enumerate}

ผลการทดสอบแสดงให้เห็นว่าระบบสามารถควบคุมการเดินได้อย่างเสถียร มีความแม่นยำสูง และพร้อมสำหรับการพัฒนาต่อยอดในอนาคต

%=============================================================================
\section{เอกสารอ้างอิง (References)}
%=============================================================================

\begin{enumerate}
    \item PyBullet Documentation. \textit{PyBullet Quickstart Guide}. Available: \url{https://pybullet.org}
    
    \item Raibert, M. H. (1986). \textit{Legged Robots That Balance}. MIT Press.
    
    \item Phase 1 Documentation: \textit{Forward Kinematics of 5-Bar Linkage}. BLEGS Project, November 2024.
    
    \item Phase 2 Documentation: \textit{Dynamic Torque Analysis}. BLEGS Project, November 2024.
    
    \item GitHub Repository: M-TRCH/BLEGS\_Analysis-Unit, Branch: Test-Phase3
\end{enumerate}

%=============================================================================
\appendix
\section{รหัสโปรแกรมที่สำคัญ (Code Appendix)}
%=============================================================================

\subsection{Main Simulation Loop}

\begin{lstlisting}[language=Python, caption=Main loop implementation]
while True:
    sim_time += time_step
    basePos, baseOrn = p.getBasePositionAndOrientation(robotId)
    
    # Warm-up logic
    if not is_walking and sim_time >= WARMUP_TIME:
        is_walking = True
        state_timer = 0.0
    
    # State machine
    if is_walking:
        state_timer += time_step
        if state_timer >= STEP_TIME:
            state_timer = 0.0
            gait_state = (gait_state + 1) % 2
    
    # Trajectory generation
    target_foot_positions_REL = {}
    if is_walking:
        # Swing/Stance logic
        ...
        # Balance control
        euler_angles = p.getEulerFromQuaternion(baseOrn)
        pitch_correction = pd_control_pitch(euler_angles[1])
        roll_correction = pd_control_roll(euler_angles[0])
    
    # IK and motor control
    for foot_link_name, target_pos in target_foot_positions_REL.items():
        # Apply corrections
        corrected_pos[0] += pitch_correction
        corrected_pos[1] += roll_correction
        
        # IK
        joint_angles = p.calculateInverseKinematics(...)
        
        # Send commands
        p.setJointMotorControlArray(...)
    
    p.stepSimulation()
    time.sleep(time_step)
\end{lstlisting}

\end{document}
