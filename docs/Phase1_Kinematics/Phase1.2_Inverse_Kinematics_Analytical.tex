\documentclass[a4paper, 12pt]{article}

% --- PACKAGES ---
\usepackage{fontspec}
\usepackage{polyglossia}
\setmainlanguage{thai}
\setotherlanguage{english}
\defaultfontfeatures{Scale=MatchLowercase}
\setmainfont{TH SarabunPSK}
\newfontfamily\thaifont{TH SarabunPSK}
\usepackage{amsmath} % For math environments
\usepackage{geometry} % For page margins
\geometry{a4paper, margin=1in}
\usepackage{graphicx}
\usepackage{hyperref}

% --- TITLE ---
\title{การวิเคราะห์ Inverse Kinematics แบบ Analytical \\ (5-Bar Parallel Linkage)}
\author{นายธีรโชติ เมืองจำนงค์}
\date{\today}

% --- DOCUMENT ---
\begin{document}

\maketitle

\section{บทนำ}
เอกสารนี้อธิบายการคำนวณ Inverse Kinematics (IK) แบบ Analytical สำหรับกลไก 5-Bar Parallel Linkage ของหุ่นยนต์ขาสองขา โดยใช้วิธีการแก้สมการทางเรขาคณิตโดยตรง ซึ่งให้ความแม่นยำสูงและรวดเร็วกว่าวิธี Numerical

\textbf{Inverse Kinematics (IK)} คือการคำนวณหามุมมอเตอร์ $\theta_A$ และ $\theta_B$ เมื่อเรารู้ตำแหน่งเป้าหมายของปลายเท้า $\mathbf{P}_F(x_f, y_f)$ ซึ่งเป็นปัญหาที่สำคัญในการควบคุมหุ่นยนต์

\subsection{ข้อได้เปรียบของวิธี Analytical}
\begin{itemize}
    \item \textbf{ความแม่นยำสูง:} ใกล้เคียง machine precision ($\sim 10^{-14}$ mm)
    \item \textbf{ความเร็ว:} คำนวณตรงไม่ต้อง iterate
    \item \textbf{Deterministic:} ได้คำตอบเดียวกันเสมอ
    \item \textbf{ไม่ต้องค่าเดาเริ่มต้น:} แต่ต้องเลือก Configuration (Elbow Up/Down)
\end{itemize}

\section{Configurations ที่เป็นไปได้}
สำหรับกลไก 5-Bar Linkage มีท่าที่เป็นไปได้ทั้งหมด 4 แบบ ขึ้นอยู่กับตำแหน่งของข้อเข่า C (ซ้าย) และ D (ขวา):

\begin{enumerate}
    \item \textbf{Config 1 (Down-Down):} C ลง, D ลง - ท่ามาตรฐาน
    \item \textbf{Config 2 (Down-Up):} C ลง, D ขึ้น
    \item \textbf{Config 3 (Up-Down):} C ขึ้น, D ลง
    \item \textbf{Config 4 (Up-Up):} C ขึ้น, D ขึ้น
\end{enumerate}

\section{วิธีการคำนวณ}

\subsection{ขั้นตอนที่ 1: หาระยะห่างระหว่างจุด}
จากสมการ Forward Kinematics:
\begin{equation}
    \mathbf{P}_F = \frac{37}{29}\mathbf{P}_E - \frac{8}{29}\mathbf{P}_D
\end{equation}

เราสามารถหาระยะห่างระหว่าง $\mathbf{P}_F$ และ $\mathbf{P}_D$:
\begin{equation}
    |\mathbf{P}_F - \mathbf{P}_D| = L_{DE} \times \frac{37}{29} = 145 \times \frac{37}{29} = 185.0 \text{ mm}
\end{equation}

\subsection{ขั้นตอนที่ 2: หาจุด D จากการตัดกันของวงกลม}
หา $\mathbf{P}_D$ จากการตัดกันของวงกลม 2 วง:
\begin{itemize}
    \item \textbf{วงที่ 1:} ศูนย์กลาง $\mathbf{P}_F$ (เป้าหมาย), รัศมี 185.0 mm
    \item \textbf{วงที่ 2:} ศูนย์กลาง $\mathbf{P}_B$ (มอเตอร์ขวา), รัศมี $L_{BD} = 105$ mm
\end{itemize}

การตัดกันของวงกลมจะได้จุด 2 จุด เราเลือกตาม Configuration ที่ต้องการ (D ลง หรือ D ขึ้น)

\subsection{ขั้นตอนที่ 3: หาจุด E}
จากสมการ:
\begin{equation}
    \mathbf{P}_E = \frac{29 \mathbf{P}_F + 8 \mathbf{P}_D}{37}
\end{equation}

\subsection{ขั้นตอนที่ 4: หาจุด C จากการตัดกันของวงกลม}
หา $\mathbf{P}_C$ จากการตัดกันของวงกลม 2 วง:
\begin{itemize}
    \item \textbf{วงที่ 1:} ศูนย์กลาง $\mathbf{P}_A$ (มอเตอร์ซ้าย), รัศมี $L_{AC} = 105$ mm
    \item \textbf{วงที่ 2:} ศูนย์กลาง $\mathbf{P}_E$, รัศมี $L_{CE} = 145$ mm
\end{itemize}

\subsection{ขั้นตอนที่ 5: คำนวณมุมมอเตอร์}
จากเวกเตอร์ที่ได้:
\begin{align}
    \theta_A &= \arctan2(y_c - y_a, x_c - x_a) \\
    \theta_B &= \arctan2(y_d - y_b, x_d - x_b)
\end{align}

\section{ผลการทดสอบ}

\subsection{กรณีทดสอบ: Home Pose}
เป้าหมาย: $\mathbf{P}_F = (0, -200)$ mm

\textbf{ตารางที่ 1:} ผลลัพธ์ IK สำหรับทั้ง 4 Configurations

\begin{table}[h]
\centering
\begin{tabular}{|l|c|c|c|c|}
\hline
\textbf{Configuration} & $\theta_A$ (\textdegree) & $\theta_B$ (\textdegree) & \textbf{Error (mm)} & \textbf{Valid} \\
\hline
Down-Down & -119.53 & -37.68 & $6.36 \times 10^{-14}$ & \checkmark \\
Down-Up   & -139.91 & -166.32 & $1.68 \times 10^{-13}$ & \checkmark \\
Up-Down   & -16.06  & -37.68 & $4.26 \times 10^{-14}$ & \checkmark \\
Up-Up     & -19.37  & -166.32 & $3.39 \times 10^{2}$ & $\times$ \\
\hline
\end{tabular}
\caption{ผลการทดสอบ IK ทั้ง 4 Configurations สำหรับเป้าหมาย (0, -200) mm}
\end{table}

\subsection{การตีความผลลัพธ์}

\textbf{คำตอบที่ถูกต้อง (Valid):} พบ 3 จาก 4 Configurations

\begin{enumerate}
    \item \textbf{Config 1 (Down-Down):} ท่ามาตรฐาน - เหมาะสมที่สุด
    \item \textbf{Config 2 (Down-Up):} ท่าแปลก - อาจเสี่ยง singularity
    \item \textbf{Config 3 (Up-Down):} ท่าแปลก - ไม่เหมาะกับการพยุงน้ำหนัก
\end{enumerate}

\textbf{Config 4 (Up-Up):} อยู่นอก Workspace (Error = 339 mm)

\section{สรุปและข้อแนะนำ}

\subsection{ข้อค้นพบสำคัญ}
\begin{itemize}
    \item กลไก 5-Bar Linkage มี \textbf{หลายคำตอบ} สำหรับจุดเดียวกัน
    \item สำหรับ $(0, -200)$ mm มีคำตอบที่ถูกต้อง \textbf{3 แบบ}
    \item ความแม่นยำระดับ $10^{-14}$ mm (machine precision)
\end{itemize}

\subsection{การนำไปใช้}
แนะนำให้ใช้ \textbf{Config 1 (Down-Down)} เพราะ:
\begin{itemize}
    \item เป็นท่าที่เสถียรและธรรมชาติ
    \item เหมาะสำหรับการพยุงน้ำหนัก
    \item หลีกเลี่ยง Singularity
\end{itemize}

\end{document}