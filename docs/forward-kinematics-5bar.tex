\documentclass[a4paper, 12pt]{article}

% --- PACKAGES ---
\usepackage{fontspec}
\usepackage{polyglossia}
\setmainlanguage{thai}
\setotherlanguage{english}
\defaultfontfeatures{Scale=MatchLowercase}
\setmainfont{TH SarabunPSK}
\newfontfamily\thaifont{TH SarabunPSK}
\usepackage{amsmath} % For math environments
\usepackage{geometry} % For page margins
\geometry{a4paper, margin=1in}
\usepackage{graphicx}
\usepackage{hyperref}

% --- TITLE ---
\title{การวิเคราะห์จลนศาสตร์ขาหุ่นยนต์ \\ (กลไก 5-Bar Linkage แบบมี Offset)}
\author{นายธีรโชติ เมืองจำนงค์}
\date{\today}

% --- DOCUMENT ---
\begin{document}

\maketitle

\section{พารามิเตอร์ทางจลนศาสตร์}
เอกสารนี้สรุปพารามิเตอร์และการคำนวณสำหรับกลไกขาหุ่นยนต์แบบ 5-Bar Parallel Linkage

\textbf{ภาพรวม:} กลไก 5-Bar Parallel Linkage คือระบบแขนกลที่ประกอบด้วย:
\begin{itemize}
    \item แขนต่อ 4 แท่ง (4 links): AC, BD, CE, DE
    \item ข้อต่อ 5 จุด: A, B, C, D, E
    \item มอเตอร์ควบคุม 2 ตัว ที่จุด A และ B
    \item ปลายเท้า (End-effector) ที่จุด F ซึ่งยื่นออกจากจุด E
\end{itemize}

กลไกนี้ใช้ในการควบคุมตำแหน่งปลายเท้าของหุ่นยนต์โดยการปรับมุมมอเตอร์ทั้งสอง

\subsection{ระบบพิกัด}
\textit{คำอธิบาย:} เราใช้ระบบพิกัด Cartesian 2D โดยตั้งจุดกำเนิดไว้กึ่งกลางระหว่างมอเตอร์ทั้งสอง เพื่อให้ระบบสมมาตรและสะดวกต่อการคำนวณ

\begin{itemize}
    \item \textbf{จุดกำเนิด:} $(0, 0)$ - กึ่งกลางระหว่างมอเตอร์
    \item \textbf{แกน X:} แนวนอน (ขวาเป็นบวก)
    \item \textbf{แกน Y:} แนวตั้ง (ขึ้นเป็นบวก)
    \item \textbf{จุด A (มอเตอร์ซ้าย):} $\mathbf{P}_A = (-42.5, 0)$ mm
    \item \textbf{จุด B (มอเตอร์ขวา):} $\mathbf{P}_B = (42.5, 0)$ mm
    \item \textbf{ระยะห่างมอเตอร์:} $|AB| = 85$ mm
\end{itemize}

\subsection{ความยาวของแขนต่อ (Link Lengths)}
\textit{คำอธิบาย:} กลไก 5-Bar Linkage ประกอบด้วยแขนต่อ 4 แท่ง ออกแบบแบบสมมาตรเพื่อความเสถียรและควบคุมง่าย

\textbf{ชั้นบน (Upper Links)} - ต่อจากมอเตอร์ไปยังข้อเข่า:
\begin{itemize}
    \item \textbf{Link AC (ซ้าย):} $L_1 = L_{AC} = 105$ mm
    \item \textbf{Link BD (ขวา):} $L_2 = L_{BD} = 105$ mm
\end{itemize}

\textbf{ชั้นล่าง (Lower Links)} - ต่อจากข้อเข่าไปยังจุด E:
\begin{itemize}
    \item \textbf{Link CE (ซ้าย):} $L_3 = L_{CE} = 145$ mm
    \item \textbf{Link DE (ขวา):} $L_4 = L_{DE} = 145$ mm
\end{itemize}

\textit{หมายเหตุ:} $L_1 = L_2$ และ $L_3 = L_4$ เพื่อความสมมาตร

\subsection{End-Effector (ปลายเท้า)}
\textit{คำอธิบาย:} ปลายเท้าของหุ่นยนต์มี 2 จุดสำคัญ:

\textbf{จุด E} - จุดตัดของแขนล่างทั้งสอง (CE และ DE):
\begin{itemize}
    \item เป็นจุดที่แขนล่างทั้งสองมาบรรจบกัน
    \item ใช้ในการคำนวณทางคณิตศาสตร์
    \item ต้องอยู่ภายใน Workspace ที่เป็นไปได้
\end{itemize}

\textbf{จุด F} - ปลายเท้าจริง (Actual End-Effector):
\begin{itemize}
    \item ยื่นออกจากจุด E ตามแนวเส้นตรง DE
    \item ระยะ Offset: $L_{EF} = 40$ mm
    \item เพิ่มพื้นที่การทำงาน (Workspace) ของหุ่นยนต์
\end{itemize}

\textbf{เงื่อนไขสำคัญ:} จุด D, E, F ต้องอยู่บนเส้นตรงเดียวกัน (collinear) เสมอ

\begin{itemize}
    \item \textbf{จุดคำนวณ:} จุด \textbf{E}
    \item \textbf{ปลายเท้าจริง:} จุด \textbf{F}
    \item \textbf{ระยะออฟเซ็ต (EF):} $L_{EF} = 40$ mm
    \item \textbf{เงื่อนไข:} จุด \textbf{D, E, F} อยู่บนเส้นตรงเดียวกัน (collinear)
\end{itemize}


\section{การหา Forward Kinematics (FK)}
วัตถุประสงค์: หาพิกัดของปลายเท้า $\mathbf{P}_F(x_f, y_f)$ จากมุมมอเตอร์ $\theta_A$ และ $\theta_B$

\textbf{Forward Kinematics (FK)} คือการคำนวณหาตำแหน่งปลายเท้าเมื่อเรารู้มุมของมอเตอร์ทั้งสอง เช่น ถ้ามอเตอร์ A หมุน 30 องศา และมอเตอร์ B หมุน 45 องศา ปลายเท้าจะอยู่ที่ไหน? นี่คือปัญหาพื้นฐานในการควบคุมหุ่นยนต์

\subsection{ขั้นตอนที่ 1: พิกัดของข้อเข่า ($P_C, P_D$)}
\textit{วิธีการ:} คำนวณตำแหน่งข้อเข่าทั้งสองข้าง (จุด C และ D) จากมุมมอเตอร์โดยใช้ตรีโกณมิติ

\textbf{มุมมอเตอร์:}
\begin{itemize}
    \item $\theta_A$ - มุมของมอเตอร์ซ้าย (จุด A) วัดจากแกน X+
    \item $\theta_B$ - มุมของมอเตอร์ขวา (จุด B) วัดจากแกน X+
    \item ทิศทางบวก: หมุนทวนเข็มนาฬิกา (CCW)
\end{itemize}

\textbf{สมการคำนวณ:}
\begin{align*}
    \mathbf{P}_A &= (-42.5, 0) \\
    \mathbf{P}_B &= (42.5, 0) \\
    \mathbf{P}_C &= \mathbf{P}_A + L_1 \begin{bmatrix} \cos \theta_A \\ \sin \theta_A \end{bmatrix} = \begin{bmatrix} -42.5 + 105 \cos \theta_A \\ 105 \sin \theta_A \end{bmatrix} \\
    \mathbf{P}_D &= \mathbf{P}_B + L_2 \begin{bmatrix} \cos \theta_B \\ \sin \theta_B \end{bmatrix} = \begin{bmatrix} 42.5 + 105 \cos \theta_B \\ 105 \sin \theta_B \end{bmatrix}
\end{align*}

\subsection{ขั้นตอนที่ 2: จุดตัดแขนล่าง ($P_E$)}
\textit{วิธีการ:} หาจุด E โดยการหาจุดตัดของวงกลมสองวง (Circle-Circle Intersection)

\textbf{วงกลมที่ 1:} ศูนย์กลางที่ $\mathbf{P}_C$, รัศมี $L_3 = 145$ mm
\begin{equation}
    |\mathbf{P}_E - \mathbf{P}_C| = L_{CE} = 145
\end{equation}

\textbf{วงกลมที่ 2:} ศูนย์กลางที่ $\mathbf{P}_D$, รัศมี $L_4 = 145$ mm
\begin{equation}
    |\mathbf{P}_E - \mathbf{P}_D| = L_{DE} = 145
\end{equation}

\textbf{ระบบสมการ:}
\begin{align*}
    (x_e - x_c)^2 + (y_e - y_c)^2 &= L_3^2 = 145^2 \\
    (x_e - x_d)^2 + (y_e - y_d)^2 &= L_4^2 = 145^2
\end{align*}

\textit{หมายเหตุ:} 
\begin{itemize}
    \item การตัดกันของวงกลมให้จุดตัด 2 จุด (Elbow Up/Down)
    \item เราเลือกจุดที่ $y_e$ น้อยกว่า (Elbow Down Configuration)
    \item ต้องแก้ระบบสมการนี้โดยใช้วิธี Geometric Method
\end{itemize}

\subsection{ขั้นตอนที่ 3: จุดปลายเท้า ($P_F$)}
\textit{วิธีการ:} คำนวณตำแหน่งปลายเท้าจริง (จุด F) ซึ่งยื่นออกจากจุด E ตามแนวเส้นตรง DE

\textbf{เงื่อนไข:}
\begin{itemize}
    \item จุด D, E, F อยู่บนเส้นตรงเดียวกัน (collinear)
    \item ระยะ $|EF| = L_{EF} = 40$ mm
    \item ทิศทางจาก D ไป E ต่อไปยัง F
\end{itemize}

\textbf{การคำนวณ:} หาทิศทางจาก D ไป E (เวกเตอร์หนึ่งหน่วย) แล้วเลื่อนจุด E ไปตามทิศทางนั้น 40 mm

เนื่องจาก $D, E, F$ อยู่บนเส้นตรงเดียวกัน:
\begin{align*}
    \vec{V}_{DE} &= \mathbf{P}_E - \mathbf{P}_D \\
    \vec{u}_{DE} &= \frac{\vec{V}_{DE}}{\| \vec{V}_{DE} \|} = \frac{\mathbf{P}_E - \mathbf{P}_D}{145} \\
    \vec{V}_{EF} &= 40 \cdot \vec{u}_{DE} = \frac{40}{145} (\mathbf{P}_E - \mathbf{P}_D) = \frac{8}{29} (\mathbf{P}_E - \mathbf{P}_D)
\end{align*}
พิกัดของ $\mathbf{P}_F$ คือ:
\begin{align*}
    \mathbf{P}_F &= \mathbf{P}_E + \vec{V}_{EF} \\
    \mathbf{P}_F &= \mathbf{P}_E + \frac{8}{29} (\mathbf{P}_E - \mathbf{P}_D) \\
    \mathbf{P}_F &= \left(1 + \frac{8}{29}\right) \mathbf{P}_E - \frac{8}{29} \mathbf{P}_D
\end{align*}
\textbf{สมการ FK สุดท้าย:}
$$
\mathbf{P}_F = \frac{37}{29} \mathbf{P}_E - \frac{8}{29} \mathbf{P}_D
$$
หรือ:
\begin{align*}
    x_f &= \frac{37}{29} x_e - \frac{8}{29} x_d \\
    y_f &= \frac{37}{29} y_e - \frac{8}{29} y_d
\end{align*}


\section{การหา Jacobian ($J_F$)}
วัตถุประสงค์: หาเมทริกซ์ $\mathbf{J}_F$ ที่เชื่อมความสัมพันธ์ $\mathbf{v}_F = \mathbf{J}_F \dot{\mathbf{q}}$ โดยที่ $\mathbf{v}_F = [\dot{x}_f, \dot{y}_f]^T$ และ $\dot{\mathbf{q}} = [\dot{\theta}_A, \dot{\theta}_B]^T$

\textbf{Jacobian Matrix} คือเครื่องมือสำคัญที่บอกว่า "ถ้าเราหมุนมอเตอร์เร็วขนาดนี้ ปลายเท้าจะเคลื่อนที่เร็วแค่ไหนในแต่ละทิศทาง" ใช้ในการควบคุมความเร็ว การวางแผนเส้นทาง และการคำนวณแรงกด

\textit{หลักการ:} Jacobian เป็นเมทริกซ์ที่แปลงความเร็วเชิงมุมของมอเตอร์ $(\dot{\theta}_A, \dot{\theta}_B)$ ให้เป็นความเร็วเชิงเส้นของปลายเท้า $(\dot{x}_f, \dot{y}_f)$

\subsection{ความสัมพันธ์ของความเร็ว}
หาอนุพันธ์ของสมการ FK สำหรับ $\mathbf{P}_F$ เทียบกับเวลา:
$$
\mathbf{v}_F = \frac{d \mathbf{P}_F}{dt} = \frac{37}{29} \frac{d \mathbf{P}_E}{dt} - \frac{8}{29} \frac{d \mathbf{P}_D}{dt}
$$
$$
\mathbf{v}_F = \frac{37}{29} \mathbf{v}_E - \frac{8}{29} \mathbf{v}_D
$$
เราต้องหา $\mathbf{v}_E = \mathbf{J}_E \dot{\mathbf{q}}$ และ $\mathbf{v}_D = \mathbf{J}_D \dot{\mathbf{q}}$

\subsection{Jacobian ของจุด D ($J_D$)}
\textit{คำอธิบาย:} จุด D เชื่อมโดยตรงกับมอเตอร์ B เท่านั้น (คอลัมน์แรกเป็น 0) ดังนั้น Jacobian จะบอกว่าการหมุนมอเตอร์ B จะทำให้จุด D เคลื่อนที่อย่างไร

จาก $\mathbf{P}_D = (42.5 + 105 \cos \theta_B, 105 \sin \theta_B)$:
$$
\mathbf{v}_D =
\begin{bmatrix} \dot{x}_d \\ \dot{y}_d \end{bmatrix}
=
\begin{bmatrix}
    \frac{\partial x_d}{\partial \theta_A} & \frac{\partial x_d}{\partial \theta_B} \\
    \frac{\partial y_d}{\partial \theta_A} & \frac{\partial y_d}{\partial \theta_B}
\end{bmatrix}
\begin{bmatrix} \dot{\theta}_A \\ \dot{\theta}_B \end{bmatrix}
$$
$$
\mathbf{J}_D =
\begin{bmatrix}
    0 & -105 \sin \theta_B \\
    0 & 105 \cos \theta_B
\end{bmatrix}
$$

\subsection{Jacobian ของจุด E ($J_E$)}
\textit{คำอธิบาย:} จุด E ซับซ้อนกว่าเพราะได้รับผลกระทบจากมอเตอร์ทั้งสองตัว ต้องใช้ implicit differentiation ของสมการวงกลมเพื่อหาว่าเมื่อข้อเข่า C และ D เคลื่อนที่ จุด E จะเคลื่อนที่อย่างไร

$\mathbf{J}_E$ ได้มาจากการหาอนุพันธ์โดยนัย (implicit differentiation) ของสมการวงกลมในหัวข้อ 2.2 ซึ่งได้ผลลัพธ์ในรูปแบบ $\mathbf{A} \mathbf{v}_E = \mathbf{B} \dot{\mathbf{q}}$:
$$
\underbrace{
\begin{bmatrix}
    x_e - x_c & y_e - y_c \\
    x_e - x_d & y_e - y_d
\end{bmatrix}
}_{\mathbf{A}}
\underbrace{
\begin{bmatrix} \dot{x}_e \\ \dot{y}_e \end{bmatrix}
}_{\mathbf{v}_E}
=
\underbrace{
\begin{bmatrix}
    B_{11} & 0 \\
    0 & B_{22}
\end{bmatrix}
}_{\mathbf{B}}
\underbrace{
\begin{bmatrix} \dot{\theta}_A \\ \dot{\theta}_B \end{bmatrix}
}_{\dot{\mathbf{q}}}
$$
โดยที่:
\begin{align*}
    B_{11} &= 105 [ (y_e - y_c) \cos \theta_A - (x_e - x_c) \sin \theta_A ] \\
    B_{22} &= 105 [ (y_e - y_d) \cos \theta_B - (x_e - x_d) \sin \theta_B ]
\end{align*}
ดังนั้น $\mathbf{J}_E = \mathbf{A}^{-1} \mathbf{B}$

\subsection{Jacobian สุดท้าย ($J_F$)}
\textit{คำอธิบาย:} เมื่อเรามี Jacobian ของจุด E และ D แล้ว เราสามารถรวมกันเพื่อหา Jacobian ของจุด F (ปลายเท้าจริง) โดยใช้สูตรที่เราหาได้ว่า $\mathbf{P}_F = \frac{37}{29} \mathbf{P}_E - \frac{8}{29} \mathbf{P}_D$ นำมาหาอนุพันธ์ตามเวลา

แทนค่า $\mathbf{J}_E$ และ $\mathbf{J}_D$ กลับเข้าไปในสมการ $\mathbf{v}_F$:
\begin{align*}
    \mathbf{v}_F &= \frac{37}{29} (\mathbf{J}_E \dot{\mathbf{q}}) - \frac{8}{29} (\mathbf{J}_D \dot{\mathbf{q}}) \\
    \mathbf{v}_F &= \left( \frac{37}{29} \mathbf{J}_E - \frac{8}{29} \mathbf{J}_D \right) \dot{\mathbf{q}}
\end{align*}
\textbf{Jacobian สุดท้าย:}
$$
\mathbf{J}_F = \frac{1}{29} (37 \mathbf{J}_E - 8 \mathbf{J}_D)
$$

\end{document}