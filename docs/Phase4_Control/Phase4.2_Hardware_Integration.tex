\documentclass[a4paper, 12pt]{article}

% --- PACKAGES ---
\usepackage{fontspec}
\usepackage{polyglossia}
\setmainlanguage{thai}
\setotherlanguage{english}
\defaultfontfeatures{Scale=MatchLowercase}
\setmainfont{TH SarabunPSK}
\newfontfamily\thaifont{TH SarabunPSK}
\newfontfamily\thaifonttt{TH SarabunPSK}
\usepackage{amsmath}
\usepackage{amssymb}
\usepackage{geometry}
\geometry{a4paper, margin=1in}
\usepackage{graphicx}
\usepackage{hyperref}
\usepackage{booktabs}
\usepackage{xcolor}
\usepackage{listings}
\usepackage{float}
\usepackage{tikz}
\usetikzlibrary{shapes,arrows,positioning,calc}

% Code listing settings
\lstset{
    basicstyle=\ttfamily\small,
    breaklines=true,
    frame=single,
    numbers=left,
    numberstyle=\tiny\color{gray},
    keywordstyle=\color{blue},
    commentstyle=\color{green!60!black},
    stringstyle=\color{red},
    showstringspaces=false
}

% Title
\title{\textbf{Phase 4.2: Hardware Integration}\\
\large การบูรณาการฮาร์ดแวร์และการทดสอบระบบจริง\\
BLEGS Analysis Unit}

\author{นายธีรโชติ เมืองจำนงค์\\
BLEGS Quadruped Robot Project}

\date{2 มกราคม 2026}

\begin{document}

\maketitle
\newpage

\tableofcontents
\newpage

%=============================================================================
\section{บทนำ (Introduction)}
%=============================================================================

\subsection{วัตถุประสงค์}
Phase 4.2 มีวัตถุประสงค์เพื่อบูรณาการระบบควบคุมที่ออกแบบใน Phase 4.1 เข้ากับฮาร์ดแวร์จริง รวมถึงการพัฒนาระบบสื่อสาร การทดสอบการทำงานของมอเตอร์ และการแก้ไขปัญหาที่พบในระหว่างการทดสอบ

\subsection{ขอบเขตการศึกษา}
\begin{itemize}
    \item การพัฒนา Binary Communication Protocol v1.1
    \item การบูรณาการ Motor Controller (MCU) กับ PC
    \item การทดสอบระบบควบคุมด้วยขาเดียว (Single Leg Testing)
    \item การแก้ไขปัญหา Motor Jitter และ Communication errors
    \item การบันทึกและวิเคราะห์ข้อมูล Motor feedback
    \item การเตรียมความพร้อมสำหรับ Quadruped scaling
\end{itemize}

\subsection{ความเชื่อมโยงกับ Phase ก่อนหน้า}
Phase 4.2 ต่อยอดจาก Phase 4.1 โดยนำระบบควบคุมที่ออกแบบไปใช้จริงบนฮาร์ดแวร์:
\begin{itemize}
    \item ใช้ Controller design และ Trajectory generator จาก Phase 4.1
    \item ใช้ Binary Protocol ที่ออกแบบใน Phase 4.1
    \item ทดสอบกับมอเตอร์ที่เลือกจาก Phase 2.2 (5 N·m BLDC)
\end{itemize}

%=============================================================================
\section{สถาปัตยกรรมระบบ (System Architecture)}
%=============================================================================

\subsection{ภาพรวมระบบ}

ระบบควบคุมหุ่นยนต์สี่ขาประกอบด้วยส่วนประกอบหลัก 3 ส่วน:

\begin{enumerate}
    \item \textbf{PC (High-Level Control):}
    \begin{itemize}
        \item Gait pattern generation
        \item Inverse Kinematics calculation
        \item Trajectory planning
        \item User interface และ Data logging
    \end{itemize}
    
    \item \textbf{Motor Controller (MCU):}
    \begin{itemize}
        \item Low-level motor control (FOC - Field Oriented Control)
        \item Position/Velocity/Current control loops
        \item Encoder reading และ State estimation
        \item Communication protocol handling
    \end{itemize}
    
    \item \textbf{BLDC Motors (8 units):}
    \begin{itemize}
        \item Actuators สำหรับขาทั้ง 4 ข้าง (2 motors/leg)
        \item Built-in encoders (14-bit magnetic encoder)
        \item Rated torque: 5 N·m
        \item Maximum speed: 300 RPM
    \end{itemize}
\end{enumerate}

\subsection{Communication Flow}

\begin{tikzpicture}[
    node distance=2cm,
    block/.style={rectangle, draw, fill=blue!20, text width=6em, text centered, rounded corners, minimum height=3em},
    arrow/.style={->, >=stealth, thick}
]
    % Nodes
    \node [block] (gait) {Gait Generator};
    \node [block, below of=gait] (ik) {Inverse Kinematics};
    \node [block, below of=ik] (protocol) {Binary Protocol};
    \node [block, right of=protocol, xshift=3cm] (mcu) {MCU Controller};
    \node [block, below of=mcu] (motor) {BLDC Motor};
    \node [block, left of=motor, xshift=-3cm] (feedback) {Data Logger};
    
    % Arrows
    \draw [arrow] (gait) -- node[right] {$P_F(t)$} (ik);
    \draw [arrow] (ik) -- node[right] {$\theta(t)$} (protocol);
    \draw [arrow] (protocol) -- node[above] {Serial} (mcu);
    \draw [arrow] (mcu) -- node[right] {PWM} (motor);
    \draw [arrow] (motor) -- node[below] {Encoder} (mcu);
    \draw [arrow] (mcu) -- node[above] {Feedback} (feedback);
    \draw [arrow] (feedback) -- (protocol);
\end{tikzpicture}

%=============================================================================
\section{Binary Communication Protocol v1.1}
%=============================================================================

\subsection{ข้อกำหนดของโปรโตคอล}

\subsubsection{ความต้องการ (Requirements)}
\begin{itemize}
    \item \textbf{Efficiency:} ส่งข้อมูลได้เร็ว (Low latency)
    \item \textbf{Reliability:} ต้องมีการตรวจสอบความถูกต้อง (Error detection)
    \item \textbf{Scalability:} รองรับมอเตอร์ได้หลายตัว (8+ motors)
    \item \textbf{Real-time:} Update rate ≥ 100 Hz
    \item \textbf{Bidirectional:} รับ-ส่งข้อมูลได้ทั้งสองทาง
\end{itemize}

\subsubsection{พารามิเตอร์การสื่อสาร}
\begin{table}[H]
\centering
\begin{tabular}{lc}
\toprule
\textbf{พารามิเตอร์} & \textbf{ค่า} \\
\midrule
Baud Rate & 921600 baud \\
Data Bits & 8 bits \\
Parity & None \\
Stop Bits & 1 bit \\
Flow Control & None \\
Packet Size (Command) & 9 bytes \\
Packet Size (Response) & 17 bytes \\
CRC Algorithm & CRC-16-CCITT \\
\bottomrule
\end{tabular}
\caption{พารามิเตอร์การสื่อสาร Serial}
\end{table}

\subsection{โครงสร้าง Packet}

\subsubsection{Command Packet (คอมพิวเตอร์ส่งไปยังไมโครคอนโทรลเลอร์)}

คำสั่งที่ส่งจาก PC ไปยัง Motor Controller มีโครงสร้างดังนี้:

\begin{table}[H]
\centering
\small
\begin{tabular}{lccp{6cm}}
\toprule
\textbf{Byte} & \textbf{Field} & \textbf{Type} & \textbf{Description} \\
\midrule
0 & Header & uint8 & 0xAA - Start marker \\
1 & Command & uint8 & Command ID (0x01-0x06) \\
2 & Motor ID & uint8 & Motor index (1-8) \\
3-6 & Position & float32 & Target position (degrees) \\
7-8 & CRC-16 & uint16 & Checksum (little-endian) \\
\bottomrule
\end{tabular}
\caption{โครงสร้าง Command Packet}
\end{table}

\textbf{หมายเหตุ:} float32 ใช้ IEEE 754 format, little-endian byte order

\subsubsection{Response Packet (ไมโครคอนโทรลเลอร์ตอบกลับไปยังคอมพิวเตอร์)}

ข้อมูล feedback ที่ส่งกลับจาก Motor Controller:

\begin{table}[H]
\centering
\small
\begin{tabular}{lccp{5.5cm}}
\toprule
\textbf{Byte} & \textbf{Field} & \textbf{Type} & \textbf{Description} \\
\midrule
0 & Header & uint8 & 0xBB - Response marker \\
1 & Motor ID & uint8 & Motor index (1-8) \\
2-5 & Current Pos & float32 & Current position (degrees) \\
6-9 & Current Vel & float32 & Current velocity (deg/s) \\
10-13 & Current & float32 & Motor current (A) \\
14 & Status & uint8 & Status flags (bit field) \\
15-16 & CRC-16 & uint16 & Checksum (little-endian) \\
\bottomrule
\end{tabular}
\caption{โครงสร้าง Response Packet}
\end{table}

\subsection{Status Flags}

Byte 14 (Status) เป็น bit field ที่บอกสถานะของมอเตอร์:

\begin{table}[H]
\centering
\begin{tabular}{clp{7cm}}
\toprule
\textbf{Bit} & \textbf{Flag} & \textbf{Description} \\
\midrule
0 & ENABLED & 1 = Motor enabled, 0 = Motor disabled \\
1 & IN\_POSITION & 1 = Reached target position (error < 1°) \\
2 & OVER\_CURRENT & 1 = Current limit exceeded \\
3 & OVER\_TEMP & 1 = Temperature warning (> 80°C) \\
4 & ENCODER\_ERROR & 1 = Encoder reading error \\
5 & COMM\_ERROR & 1 = Communication error (CRC mismatch) \\
6-7 & RESERVED & Reserved for future use \\
\bottomrule
\end{tabular}
\caption{Status Flags Definition}
\end{table}

\subsection{Command Types}

\begin{table}[H]
\centering
\begin{tabular}{clp{7cm}}
\toprule
\textbf{ID} & \textbf{Command} & \textbf{Description} \\
\midrule
0x01 & SET\_POSITION & ส่งคำสั่งตำแหน่งเป้าหมายไปยังมอเตอร์ \\
0x02 & GET\_FEEDBACK & ขอข้อมูล feedback จากมอเตอร์ (position, velocity, current) \\
0x03 & ENABLE\_MOTOR & เปิดใช้งานมอเตอร์ (Enable torque output) \\
0x04 & DISABLE\_MOTOR & ปิดใช้งานมอเตอร์ (Disable torque, free-wheel) \\
0x05 & SET\_ZERO & ตั้งตำแหน่งปัจจุบันเป็นศูนย์ (Calibration) \\
0x06 & EMERGENCY\_STOP & หยุดฉุกเฉิน (Stop all motors immediately) \\
\bottomrule
\end{tabular}
\caption{รายการ Command Types}
\end{table}

\subsection{การจัดการ Error}

\subsubsection{CRC-16 Checksum}

ใช้ CRC-16-CCITT (Polynomial 0x1021) เพื่อตรวจจับความผิดพลาด:

\begin{lstlisting}[language=Python, caption=CRC-16 Calculation]
def crc16_ccitt(data):
    """Calculate CRC-16-CCITT checksum"""
    crc = 0xFFFF
    polynomial = 0x1021
    
    for byte in data:
        crc ^= (byte << 8)
        for _ in range(8):
            if crc & 0x8000:
                crc = ((crc << 1) ^ polynomial) & 0xFFFF
            else:
                crc = (crc << 1) & 0xFFFF
    
    return crc
\end{lstlisting}

\subsubsection{Timeout Protection}

เมื่อไม่ได้รับ Response ภายใน Timeout period ระบบจะดำเนินการดังนี้:

\begin{itemize}
    \item \textbf{Timeout period:} 100 ms เป็นเวลาสูงสุดที่รอก่อนจะถือว่าไม่ได้รับ Response
    \item \textbf{Action:} ทำการ Retry สูงสุด 3 ครั้ง โดยมีการหน่วงเวลา 10 ms ระหว่างแต่ละครั้ง
    \item \textbf{Fallback:} ถ้า retry ครบ 3 ครั้งแล้วยังไม่สำเร็จ ระบบจะสั่ง Emergency stop เพื่อความปลอดภัย
\end{itemize}

\subsubsection{Packet Loss Recovery}

ขั้นตอนการกู้คืนเมื่อเกิดการสูญหาย Packet:

\begin{enumerate}
    \item \textbf{ตรวจสอบ CRC:} เมื่อได้รับ Packet ใหม่ ตรวจสอบ CRC ก่อนดำเนินการต่อ
    \item \textbf{จัดการ CRC Mismatch:} ถ้า CRC ไม่ตรงกัน ให้ทิ้ง Packet ที่เสียหายและขอส่งใหม่
    \item \textbf{บันทึก Error:} เก็บข้อมูล Error rate สำหรับการวิเคราะห์และปรับปรุงระบบ
\end{enumerate}

%=============================================================================
\section{การทดสอบขาเดียว (Single Leg Testing)}
%=============================================================================

\subsection{การติดตั้งฮาร์ดแวร์}

\subsubsection{อุปกรณ์ที่ใช้}
\begin{itemize}
    \item \textbf{Motor:} 2 units BLDC 5 N·m (Thigh motor + Shank motor)
    \item \textbf{Motor Controller:} Custom MCU board (STM32-based)
    \item \textbf{Power Supply:} 24V DC, 10A
    \item \textbf{PC:} Python control software
    \item \textbf{USB-to-Serial:} CH340G @ 921600 baud
\end{itemize}

\subsubsection{การเชื่อมต่อ}
\begin{verbatim}
PC (Python) <--USB--> USB-to-Serial <--UART--> MCU <--3-phase PWM--> Motors
\end{verbatim}

\subsection{ขั้นตอนการทดสอบ}

\subsubsection{Phase 1: Hardware Verification}
\begin{enumerate}
    \item ตรวจสอบการเชื่อมต่อสาย Power และ Signal
    \item ทดสอบ Serial communication (Echo test)
    \item Calibrate มอเตอร์ (SET\_ZERO command)
    \item ทดสอบคำสั่งพื้นฐาน (ENABLE, DISABLE, SET\_POSITION)
\end{enumerate}

\subsubsection{Phase 2: Position Control Testing}
\begin{enumerate}
    \item ส่งคำสั่งตำแหน่งคงที่ (Static position)
    \item ทดสอบความแม่นยำ (Position error measurement)
    \item ทดสอบ Repeatability (10 รอบ)
    \item บันทึกข้อมูล feedback
\end{enumerate}

\subsubsection{Phase 3: Trajectory Tracking}
\begin{enumerate}
    \item สร้าง Simple sinusoidal trajectory
    \item ทดสอบ Trajectory tracking @ 50 Hz
    \item เพิ่ม Update rate เป็น 100 Hz
    \item วิเคราะห์ Tracking error
\end{enumerate}

\subsubsection{Phase 4: Gait Pattern Testing}
\begin{enumerate}
    \item ใช้ Elliptical trajectory (60×30 mm)
    \item ทดสอบ 1 cycle (600 ms)
    \item ทดสอบ Continuous gait (341+ cycles)
    \item บันทึกข้อมูล Motor feedback ทุกรอบ
\end{enumerate}

\subsection{ผลการทดสอบ}

\subsubsection{Hardware Verification Results}

\begin{table}[H]
\centering
\begin{tabular}{lcc}
\toprule
\textbf{Test Item} & \textbf{Result} & \textbf{Status} \\
\midrule
Serial Communication & 921600 baud stable & \checkmark PASS \\
CRC Error Rate & $< 0.1$\% & \checkmark PASS \\
Motor Calibration & Zero position set & \checkmark PASS \\
Enable/Disable & Responsive & \checkmark PASS \\
Emergency Stop & < 50 ms response & \checkmark PASS \\
\bottomrule
\end{tabular}
\caption{ผลการทดสอบฮาร์ดแวร์พื้นฐาน}
\end{table}

\subsubsection{Position Control Accuracy}

\begin{table}[H]
\centering
\begin{tabular}{lcc}
\toprule
\textbf{Metric} & \textbf{Motor A} & \textbf{Motor B} \\
\midrule
Mean Error & $0.8$ deg & $0.9$ deg \\
RMS Error & $1.5$ deg & $1.7$ deg \\
Max Error & $4.2$ deg & $4.8$ deg \\
Repeatability & $\pm 0.5$ deg & $\pm 0.6$ deg \\
Settling Time & $80$ ms & $75$ ms \\
\bottomrule
\end{tabular}
\caption{ความแม่นยำการควบคุมตำแหน่ง}
\end{table}

\subsubsection{Gait Pattern Testing Results}

\begin{table}[H]
\centering
\begin{tabular}{lc}
\toprule
\textbf{Test Parameter} & \textbf{Result} \\
\midrule
Total Test Duration & 205 seconds \\
Total Cycles Completed & 341+ cycles \\
Successful Cycles & 327-337 cycles \\
Success Rate & 96-99\% \\
Average Cycle Time & 601 ms (target: 600 ms) \\
Communication Errors & $< 1$\% \\
Position Error (RMS) & $< 2$ degrees \\
Maximum Current Draw & 3.2 A (peak) \\
Average Current Draw & 1.1 A \\
Motor Temperature & 45-52°C \\
\midrule
\textbf{Overall Status} & \checkmark \textbf{PASS} \\
\bottomrule
\end{tabular}
\caption{ผลการทดสอบ Gait Pattern ต่อเนื่อง}
\end{table}

%=============================================================================
\section{ปัญหาและการแก้ไข (Problems and Solutions)}
%=============================================================================

\subsection{ปัญหา Motor Jitter}

\subsubsection{ลักษณะอาการ}
\begin{itemize}
    \item มอเตอร์สั่นเล็กน้อยเมื่อรับคำสั่งตำแหน่งใหม่
    \item เสียงดัง (ความถี่สูง ~200 Hz)
    \item Current draw เพิ่มขึ้นแบบ spike
    \item อุณหภูมิมอเตอร์สูงขึ้น (~10°C)
\end{itemize}

\subsubsection{การวินิจฉัยสาเหตุ}

\textbf{ทดสอบที่ 1: ลด Update Rate}

ทำการทดสอบโดยปรับ Update rate เป็น 3 ระดับและสังเกตผลกระทบต่อ Jitter:
\begin{itemize}
    \item ทดสอบ 50 Hz: Jitter ลดลง 50\% แต่ความเร็วในการอัพเดตตำแหน่งต่ำเกินไป
    \item ทดสอบ 100 Hz: Jitter ลดลง 30\% เป็นค่าที่สมดุลระหว่างความเร็วและความนุ่มนวล
    \item ทดสอบ 200 Hz: Jitter รุนแรงมาก มอเตอร์สั่นอย่างรุนแรง
\end{itemize}

\textbf{สรุปการวินิจฉัย:} Update rate ที่สูงเกินไปทำให้มอเตอร์ไม่ทันตอบสนองต่อคำสั่งตำแหน่งใหม่ที่เข้ามาอย่างรวดเร็ว ทำให้เกิด Oscillation ระหว่างตำแหน่งเป้าหมายและตำแหน่งจริง

\textbf{ทดสอบที่ 2: วิเคราะห์ Trajectory}

ทำการวิเคราะห์ Trajectory ที่สร้างขึ้นพบว่า:
\begin{itemize}
    \item Trajectory มี Discontinuity (กระโดด) ในบางจุด
    \item ความเร่งเปลี่ยนแปลงแบบ step change ที่เปลี่ยนกะทันหัน
    \item Jerk สูง (คืออัตราการเปลี่ยนแปลงของความเร่ง) ทำให้การเคลื่อนที่ไม่นุ่มนวล
\end{itemize}

\textbf{สรุปการวินิจฉัย:} Trajectory ที่ไม่นุ่มนวลพอทำให้มอเตอร์พยายามติดตามการเปลี่ยนแปลงตำแหน่งที่กระโดด ซึ่งเกินความสามารถของมอเตอร์ ทำให้เกิด Jitter

\subsubsection{วิธีแก้ไข}

\textbf{Solution 1: ปรับ Update Rate}

ลด Update rate จาก 200 Hz ลงเป็น 100 Hz ซึ่งเป็นค่าที่เหมาะสมกับความสามารถในการตอบสนองของมอเตอร์

\textbf{ผลลัพธ์:} Jitter ลดลงประมาณ 30\%

\textbf{Solution 2: Trajectory Smoothing}
\begin{lstlisting}[language=Python, caption=Trajectory Smoothing]
def smooth_trajectory(trajectory, window_size=5):
    """Apply moving average filter"""
    smoothed = np.convolve(
        trajectory, 
        np.ones(window_size)/window_size, 
        mode='same'
    )
    return smoothed

# Apply smoothing
x_smooth = smooth_trajectory(x_trajectory, window_size=5)
y_smooth = smooth_trajectory(y_trajectory, window_size=5)
\end{lstlisting}

ผล: Jitter ลดลง ~40\%

\textbf{Solution 3: Motor Parameter Tuning (MCU Firmware)}

ปรับจูนพารามิเตอร์ PID controller ในมอเตอร์:

\begin{table}[H]
\centering
\begin{tabular}{lccc}
\toprule
\textbf{Parameter} & \textbf{Before} & \textbf{After} & \textbf{Change} \\
\midrule
Position P-Gain & 50 & 30 & -40\% \\
Position I-Gain & 10 & 5 & -50\% \\
Position D-Gain & 20 & 15 & -25\% \\
Velocity Feedforward & 0.0 & 0.3 & +0.3 \\
Current Limit & 10 A & 8 A & -20\% \\
\bottomrule
\end{tabular}
\caption{การปรับจูนพารามิเตอร์มอเตอร์}
\end{table}

ผล: Jitter ลดลง ~60\%

\subsubsection{ผลลัพธ์รวม}

หลังจากใช้ทั้ง 3 solutions ร่วมกัน พบว่าปัญหา Motor Jitter ลดลงอย่างมาก:
\begin{itemize}
    \item Jitter ลดลงประมาณ 90\% มอเตอร์ทำงานได้เสถียรมากขึ้น
    \item เสียงเบาลงมาก แทบไม่ได้ยินเสียงสั่นความถี่สูง
    \item Current draw ลดลงประมาณ 25\% ทำให้ประหยัดพลังงานได้
    \item อุณหภูมิมอเตอร์ลดลง 8-10°C ซึ่งอยู่ในเกณฑ์ที่ปลอดภัย
    \item Success rate เพิ่มขึ้นจาก 85\% เป็น 96-99\% ซึ่งเกินเป้าหมายที่ตั้งไว้
\end{itemize}

\subsection{ปัญหา Communication Timeout}

\subsubsection{ลักษณะอาการ}
\begin{itemize}
    \item บางครั้งไม่ได้รับ Response packet จากมอเตอร์
    \item เกิด Timeout error ~3-5\% ของ commands
    \item มอเตอร์หยุดเคลื่อนที่ชั่วคราว
\end{itemize}

\subsubsection{สาเหตุ}
\begin{enumerate}
    \item \textbf{Serial Buffer Overflow:} MCU ไม่ทันประมวลผล packet
    \item \textbf{CRC Mismatch:} Noise บนสาย serial
    \item \textbf{Busy MCU:} MCU ทำงานหนักเกินไป (Motor control + Communication)
\end{enumerate}

\subsubsection{วิธีแก้ไข}

\textbf{Solution 1: Increase Serial Buffer Size (MCU)}

เพิ่มขนาด Buffer ของ MCU เพื่อรองรับการรับส่งข้อมูลจำนวนมาก:
\begin{itemize}
    \item เพิ่ม RX buffer จาก 64 bytes เป็น 256 bytes (4 เท่า)
    \item เพิ่ม TX buffer จาก 64 bytes เป็น 256 bytes (4 เท่า)
\end{itemize}

การเพิ่ม Buffer ทำให้ MCU มีพื้นที่ในการเก็บข้อมูลมากขึ้น ลดความเสี่ยงของ Buffer overflow

\textbf{Solution 2: Optimize MCU Firmware}
\begin{itemize}
    \item ใช้ DMA (Direct Memory Access) สำหรับ Serial communication
    \item ลด Priority ของ Communication interrupt
    \item เพิ่ม Priority ของ Motor control interrupt
\end{itemize}

\textbf{Solution 3: Add Retry Logic (PC Side)}
\begin{lstlisting}[language=Python, caption=Retry Logic]
def send_command_with_retry(motor_id, position, max_retries=3):
    """Send command with automatic retry"""
    for attempt in range(max_retries):
        try:
            response = send_command(motor_id, position)
            if verify_crc(response):
                return response
        except TimeoutError:
            print(f"Retry {attempt+1}/{max_retries}")
            time.sleep(0.01)  # 10ms delay
    
    raise CommunicationError("Max retries exceeded")
\end{lstlisting}

\subsubsection{ผลลัพธ์}

หลังจากแก้ปัญหา Communication Timeout พบว่าประสิทธิภาพการสื่อสารดีขึ้นอย่างมาก:
\begin{itemize}
    \item Timeout rate ลดลงจาก 3-5\% เป็นน้อยกว่า 1\% ซึ่งอยู่ในเกณฑ์ที่ยอมรับได้
    \item Communication เสถียรขึ้น ไม่พบปัญหาการขาดการสื่อสารในระหว่างทดสอบ
    \item ไม่พบ Buffer overflow อีกต่อไป
\end{itemize}

%=============================================================================
\section{Data Logging และการวิเคราะห์}
%=============================================================================

\subsection{ระบบบันทึกข้อมูล}

\subsubsection{ข้อมูลที่บันทึก}
\begin{itemize}
    \item \textbf{Timestamp:} เวลา (ms) ตั้งแต่เริ่มทดสอบ
    \item \textbf{Motor ID:} หมายเลขมอเตอร์ (1-8)
    \item \textbf{Target Position:} ตำแหน่งเป้าหมาย (degrees)
    \item \textbf{Actual Position:} ตำแหน่งจริงจาก Encoder (degrees)
    \item \textbf{Velocity:} ความเร็วของมอเตอร์ (deg/s)
    \item \textbf{Current:} กระแสไฟฟ้า (A)
    \item \textbf{Status Flags:} สถานะของมอเตอร์
\end{itemize}

\subsubsection{รูปแบบไฟล์ CSV}

\begin{lstlisting}[caption=ตัวอย่างไฟล์ motor\_feedback\_20260101\_175601.csv]
timestamp,motor_id,target_pos,actual_pos,velocity,current,status
0.010,1,-45.23,-45.18,2.3,0.8,0x03
0.020,2,-120.56,-120.51,-1.2,0.6,0x03
0.030,1,-45.67,-45.62,3.1,0.9,0x03
...
\end{lstlisting}

\subsection{การวิเคราะห์ข้อมูล}

\subsubsection{Script วิเคราะห์}

มี Python script สำหรับวิเคราะห์ข้อมูล: \texttt{scripts/analysis/Plot\_Motor\_Log.py}

\textbf{คุณสมบัติ:}
\begin{itemize}
    \item Plot Position tracking (Target vs Actual)
    \item Plot Position error over time
    \item Plot Velocity profile
    \item Plot Current consumption
    \item คำนวณ Statistics (Mean, RMS, Max error)
\end{itemize}

\subsubsection{ผลการวิเคราะห์}

\textbf{Position Tracking Performance:}
\begin{itemize}
    \item Mean error: 0.8-0.9 degrees
    \item RMS error: 1.5-1.7 degrees
    \item Max error: 4.2-4.8 degrees
    \item Settling time: 75-80 ms
\end{itemize}

\textbf{Power Consumption:}
\begin{itemize}
    \item Average current: 1.1 A (per motor)
    \item Peak current: 3.2 A (during acceleration)
    \item Average power: 26.4 W (@ 24V)
    \item Total power (2 motors): ~50-60 W
\end{itemize}

\textbf{Thermal Performance:}
\begin{itemize}
    \item Steady-state temperature: 45-52°C
    \item Ambient temperature: 25-28°C
    \item Temperature rise: ~20-25°C
    \item Thermal limit: 80°C (Warning threshold)
\end{itemize}

%=============================================================================
\section{การเตรียมความพร้อมสำหรับ Quadruped}
%=============================================================================

\subsection{ข้อควรพิจารณา}

\subsubsection{Scaling Challenges}
เมื่อขยายจาก 1 ขา (2 motors) เป็น 4 ขา (8 motors):

\begin{enumerate}
    \item \textbf{Communication Bandwidth:}
    \begin{itemize}
        \item 1 ขา @ 100 Hz = 100 commands/s × 2 motors = 200 packets/s
        \item 4 ขา @ 100 Hz = 800 packets/s
        \item Bandwidth required: 800 × (9+17) bytes = 20.8 KB/s
        \item Available @ 921600 baud: ~115 KB/s
        \item Margin: ~80\% (เพียงพอ)
    \end{itemize}
    
    \item \textbf{Processing Power (PC):}
    \begin{itemize}
        \item IK calculation: 8 motors × 100 Hz = 800 IK/s
        \item คำนวณเวลา: ~0.1 ms/IK
        \item รวม: ~80 ms/s (8\% CPU load)
    \end{itemize}
    
    \item \textbf{Synchronization:}
    \begin{itemize}
        \item ต้องส่งคำสั่งไปยัง 8 motors พร้อมกัน
        \item Latency ต้องไม่เกิน 1-2 ms ระหว่างมอเตอร์
        \item ต้องใช้ Multi-threading หรือ Async I/O
    \end{itemize}
\end{enumerate}

\subsection{ข้อเสนอแนะสำหรับ Phase 5}

\subsubsection{Motor Indexing System}
\begin{table}[H]
\centering
\begin{tabular}{cccc}
\toprule
\textbf{Leg} & \textbf{Position} & \textbf{Thigh Motor} & \textbf{Shank Motor} \\
\midrule
FL & Front-Left & Motor 1 & Motor 2 \\
FR & Front-Right & Motor 3 & Motor 4 \\
RL & Rear-Left & Motor 5 & Motor 6 \\
RR & Rear-Right & Motor 7 & Motor 8 \\
\bottomrule
\end{tabular}
\caption{Motor Indexing Scheme}
\end{table}

\subsubsection{Mirror Kinematics}

ขาซ้ายและขาขวามี Kinematics สมมาตร:
\begin{itemize}
    \item \textbf{Left legs (FL, RL):} Motor A ที่ $x = -42.5$ mm, Motor B ที่ $x = +42.5$ mm
    \item \textbf{Right legs (FR, RR):} Motor A ที่ $x = +42.5$ mm, Motor B ที่ $x = -42.5$ mm (Mirrored)
\end{itemize}

Trajectory transformation:
\begin{equation}
    \mathbf{P}_{F,right}(x, y) = \mathbf{P}_{F,left}(-x, y)
\end{equation}

\subsubsection{Multi-Motor Communication}

ใช้ Threading สำหรับ Parallel communication:

\begin{lstlisting}[language=Python, caption=Multi-Motor Control]
import threading

def control_motor_pair(motor_pair_id, trajectory):
    """Control 2 motors (1 leg) in parallel"""
    motor_A_id = motor_pair_id * 2 + 1
    motor_B_id = motor_pair_id * 2 + 2
    
    for point in trajectory:
        theta_A, theta_B = calculate_ik(point)
        send_command(motor_A_id, theta_A)
        send_command(motor_B_id, theta_B)
        time.sleep(0.01)  # 100 Hz

# Create 4 threads (1 per leg)
threads = []
for leg_id in range(4):
    t = threading.Thread(
        target=control_motor_pair,
        args=(leg_id, trajectory[leg_id])
    )
    threads.append(t)
    t.start()

# Wait for all threads
for t in threads:
    t.join()
\end{lstlisting}

%=============================================================================
\section{สรุปและข้อเสนอแนะ (Conclusion and Recommendations)}
%=============================================================================

\subsection{สรุปผลการพัฒนา}

Phase 4.2 ประสบความสำเร็จในการบูรณาการฮาร์ดแวร์และระบบควบคุม:

\begin{itemize}
    \item[\checkmark] \textbf{Binary Protocol v1.1:} พัฒนาและทดสอบสำเร็จ (Success rate 96-99\%)
    \item[\checkmark] \textbf{Single Leg Testing:} ทดสอบ 341+ cycles ได้สำเร็จ
    \item[\checkmark] \textbf{Motor Jitter:} แก้ไขและลดลง ~90\%
    \item[\checkmark] \textbf{Communication Stability:} Error rate $< 1$\%
    \item[\checkmark] \textbf{Data Logging:} ระบบบันทึกและวิเคราะห์ข้อมูลสมบูรณ์
\end{itemize}

\subsection{ผลการทดสอบโดยรวม}

\begin{table}[H]
\centering
\begin{tabular}{lcc}
\toprule
\textbf{Metric} & \textbf{Target} & \textbf{Achieved} \\
\midrule
Position Accuracy (RMS) & $< 3$ deg & $< 2$ deg \\
Update Rate & $\geq 50$ Hz & 100 Hz \\
Success Rate & $> 90$\% & 96-99\% \\
Communication Error & $< 5$\% & $< 1$\% \\
Motor Temperature & $< 80$°C & 45-52°C \\
Current Draw & $< 5$ A & 3.2 A (peak) \\
\midrule
\textbf{Overall Status} & & \checkmark \textbf{PASS} \\
\bottomrule
\end{tabular}
\caption{สรุปผลการทดสอบเทียบกับเป้าหมาย}
\end{table}

\subsection{บทเรียนที่ได้รับ}

\begin{enumerate}
    \item \textbf{Update Rate Optimization:}
    \begin{itemize}
        \item 100 Hz เหมาะสมที่สุด (Balance ระหว่างความเร็วและเสถียรภาพ)
        \item 200 Hz สูงเกินไป (เกิด Motor jitter)
        \item 50 Hz ต่ำเกินไป (Trajectory tracking ไม่ดี)
    \end{itemize}
    
    \item \textbf{Trajectory Smoothing มีความสำคัญ:}
    \begin{itemize}
        \item Moving average filter ช่วยลด Jitter ได้ ~40\%
        \item S-Curve profiling ลด Jerk และแรงกระแทก
        \item Continuous trajectory ดีกว่า Discrete waypoints
    \end{itemize}
    
    \item \textbf{Motor Parameter Tuning:}
    \begin{itemize}
        \item Position P-gain ต้องไม่สูงเกินไป (เกิด Oscillation)
        \item Velocity Feedforward ช่วยลด Tracking error
        \item Current limit ป้องกัน Overshoot
    \end{itemize}
    
    \item \textbf{Communication Reliability:}
    \begin{itemize}
        \item CRC-16 จำเป็นมาก (ตรวจจับ Error ได้)
        \item Retry logic ช่วยเพิ่มความเสถียร
        \item DMA และ Buffer size มีผลต่อประสิทธิภาพ
    \end{itemize}
\end{enumerate}

\subsection{ข้อเสนอแนะสำหรับการพัฒนาต่อไป}

\subsubsection{Phase 5.1: Quadruped Scaling}
\begin{itemize}
    \item พัฒนา Motor indexing system (8 motors)
    \item ออกแบบ Mirror kinematics สำหรับขาซ้าย-ขวา
    \item ทดสอบ Multi-motor synchronization
    \item พัฒนา Gait pattern coordination
\end{itemize}

\subsubsection{Phase 5.2: Gait Tuning \& Optimization}
\begin{itemize}
    \item ทดสอบ Gait modes ต่างๆ (Trot, Walk, Crawl)
    \item พัฒนา Asymmetric trajectory generation
    \item ปรับจูนพารามิเตอร์ (Step length, Lift height, Cycle time)
    \item ทดสอบการเดินบนพื้นจริง
\end{itemize}

\subsubsection{Phase 6: Sensor Feedback (Future Work)}
\begin{itemize}
    \item ติดตั้ง IMU sensor (BNO086)
    \item พัฒนา Balance controller ด้วย PD control
    \item ชดเชยท่าทางการเดินด้วย Sensor feedback
    \item ทดสอบบนพื้นเอียงและพื้นไม่เรียบ
\end{itemize}

%=============================================================================
\section{เอกสารอ้างอิง (References)}
%=============================================================================

\begin{enumerate}
    \item Phase 4.1: Controller Design - BLEGS Analysis Unit
    \item Phase 3.1: Gait Control Simulation - BLEGS Analysis Unit
    \item Binary Communication Protocols for Embedded Systems
    \item CRC-16-CCITT Checksum Standard (ITU-T Recommendation V.41)
    \item BLDC Motor Control and FOC Algorithms
    \item Real-time Serial Communication Best Practices
    \item Motor Control Parameter Tuning Guidelines
    \item Data Logging and Analysis for Robotics Systems
\end{enumerate}

\end{document}
